\subsection{Error and convergence study}
\label{sec:analysis}

In the current work, we employ a tensorized Gaussian quadrature to construct the PC surrogate
model. To this end, 125 \geoclaw simulations were used to compute the PC coefficients using Equation~\ref{eq:nisp}.
The computational cost of running this ensemble is low compared to the cost of the thousands of 
runs of the forward model that would have been required if a standard Monte-Carlo uncertainty propagation technique was to be adopted.  Figure~\ref{fig:rlzs} plots the evolution of the
water surface elevation predicted at the four different gauges 
using \geoclaw for the 125 different realizations 
required to compute the PC expansions. We notice that the 
variability in water surface elevation is insignificant in the first 
hour at all gauges as the plots of the different realizations superimpose.
Later in time, the variability starts to increase for all gauges 
as indicated by the thickness of the bands formed by the plots of curves for the selected realizations.
This variability appears to be significant at $t\sim$ 5400~s, 6000~s, 3600~s and 7200~s
for gauges 21401, 21413, 21418, 21419, respectively.
This is consistent with the distance from the gauge to the epicenter of the earthquake located approximately 72 kilometer east of Japan where gauge 21418 is the closest and gauge 21419 is the farthest as shown in Figure~\ref{fig:setup_buoy_locations}.
The uncertainty in the prediction of water surface  elevation persists till the end of the simulations
at all gauges.

In order to check the consistency of the PC approximation, we compare
water surface elevation from the realizations 
with those obtained from the PC surrogate. The different curves (not shown) 
reveal excellent agreement for all times and gauge locations.
To quantify this agreement, we define
an error metric that measures the relative normalized root mean-square error between the left hand side function 
in Equation~\eqref{eq:stochseries} and its PC representation at the sampling points:

\begin{equation} 
   E = \frac{\displaystyle
         \left(\sum_{\xxi \in \NISP} \left|U(\xxi) - \sum_{k = 0}^{P}
U_k\Psi_k(\xxi)\right|^2
         \right)^{1/2}}
        {\displaystyle
          \left(\sum_{\xxi \in \NISP} \left|U(\xxi)\right|^2\right)^{1/2} 
          },
\label{eq:error}
\end{equation}
where $\NISP$ is the 125-member ensemble obtained to construct the PC surrogate. 
This error metric calculated at the different gauge locations, is shown in Figure~\ref{fig:error};
the largest relative normalized error for 
water surface elevation is about 0.1\%. 

Another check for the validity of the PC approximation 
consists of verifying whether the probability density
functions (pdfs) of water surface elevation at the different gauge locations
converges with increased order of the PC expansion~\cite{Alexanderian2012,sraj:2013a}.  Sample
water surface elevation pdfs are shown in Figure~\ref{fig:pdfs2}
and Figure~\ref{fig:pdfs3} where
the different curves correspond to increased order of PC ($p= 1-5$).
The plots indicate double peaked distributions that are
well-resolved with PC order $p=3$ at t = 7200 s as shown in Figure~\ref{fig:pdfs2}.
However, at t = 10800 s further refinement is needed as the pdfs are
sensitive to the refinement up to order $p=4$ but then becomes weakly 
insensitive with additional refinement  ($p = 5$) as shown in Figure~\ref{fig:pdfs3}. 
We therefore used PC order $p = 5$ in all computations below.
This test and the various error metrics presented above provide confidence that the PC expansion is a faithful 
model surrogate that can be used in both the forward and inverse problems. 

