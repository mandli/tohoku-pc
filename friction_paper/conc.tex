%!TEX root = paper.tex
\section{Discussion and Conclusions}
\label{sec:conc}

The present study aimed at estimating Manning's $n$ friction coefficient  which
plays an important role in the accurate prediction of water surface elevation in
tsunami modeling. We proposed a three-parameter representation of Manning's $n$
friction coefficient that was characterized using iso-baths  to define three
distinct regions, on-shore, near-shore, and deep-water, in the domain and
represented each region with a single Manning's $n$ coefficient.  The estimation
relied on a Bayesian inference approach that sharpens the initial estimates of
the three uncertain parameters using real observations.

In our test case, the \tohoku tsunami, we used four DART buoy gauges that
provide  surface elevation information.  To accelerate the Bayesian inference,
we relied on the polynomial  chaos expansions to produce a faithful and
efficient surrogate of the forward model \geoclaw.  This PC surrogate model was
then used to quantify the uncertainties in the predicted water surface
elevations due to the uncertainties in the Manning's $n$ coefficient.  This
included the mean and standard deviation of water surface elevations which were
also compared to the measured buoy data.  A global sensitivity analysis was also
performed in order to quantify the contribution of each uncertain parameter to
the variance in surface elevation.  It was found that the Manning's $n$
coefficient in the near-shore region $n_2$ contributed the most to the variance
in the surface elevations compared to the other two Manning's $n$ coefficients
in the on-shore and deep-water regions.  This is expected as the deep-water
friction has little effect on the overall column.  On the other hand, the
sensitivity of the analysis to the on-shore value being less than that in the
deep-water was unexpected. We suspect that this is primarily due to the use of the
DART buoy gauges, being far enough from land regions to have a significant impact. 
This might not be the case if data from inundation zones (that
were not available to us) were used. 

Using Bayesian inference, MAP estimates were found for the three region's $n$
parameters using MCMC.  These values were (excluding $n_1$ where no meaningful
MAP value was found), $n_2=0.011$ and $n_3=0.180$.  The values and their
corresponding marginalized PDFs (Figure~\ref{fig:pdfs}) tell perhaps the most
complete story in this analysis.  As was mentioned above, $n_2$ is the most
sensitive and has the most clear MAP value.  Although the value was lower than
expected, the tail includes the most commonly used values (0.022-0.025).  The
deep-water value $n_3$ has a peak that is much higher than expected but a tail
that does not taper off particularly quickly.  This is not altogether unexpected
due to the low sensitivity of the simulations to this region's friction.  The
on-shore value $n_1$ shows again how insensitive the used simulations and
observations were to on-shore values in our particular setting.  
In conclusion, it is clear that the use of only the DART buoy observations and the 
bathymetry resolution used here is not sufficiently informative to invert for the 
Manning's $n$ values with any strong confidence, especially outside of the near-shore region.

Finally, it should be noted that the friction source term is only one of a
number of other sources of uncertainty.  In particular, the rupture model and
bathymetry might often have an important influence on the tsunami.  
In the future, this type of analysis would be more illustrative if
propagation of uncertainties in both the bathymetry and source models were
included in each Manning's $n$ estimate.  The
present study focused on formulating and estimating a low-dimensional
representation of the Manning's $n$ coefficient using UQ techniques, namely
Bayesian inference and PC expansions.  A high-dimensional representation of the
Manning's $n$ coefficient would require a large number of forward runs and this remains
computationally expensive.  We will work on alternative methods to reduce the
dimensionality of the problem and this will be the objective of a future study.
