%&pdflatex
%
%
%
%

% \documentclass[12pt]{article}
\documentclass[preprint,12pt]{elsarticle}

\usepackage{graphicx}
\usepackage{bm}
\usepackage{natbib}
\usepackage{xspace} % Incorporates correct optional space after commands
\usepackage{hyperref}

% Add line numbers
\usepackage[mathlines]{lineno} % Control line numbers

% Multipart figures
\usepackage{subcaption}  % "Newest" subfigure like package

% \oddsidemargin 0in
% \evensidemargin 0in
% \textwidth= 6.5in
% \topmargin -0.50in
% \textheight 9in

\usepackage{ifpdf} % For declaring rules for graphics below

\usepackage{latexsym,amssymb,amsmath,color}
\newcommand{\revised}[1]{\textcolor{blue}{{#1}}}
\newcommand{\comment}[1]{\textcolor{red}{{#1}}}
\newcommand{\alert}[1]{\textbf{\textcolor{red}{{#1}}}}
\newcommand{\be}{\begin{equation}}
\newcommand{\ee}{\end{equation}}
\newcommand{\xxi}{\vec{\xi}}
\newcommand{\xxii}{\xxi^{(\scriptscriptstyle{-}\scriptstyle{i})}}
\newcommand{\var}[1]{{\mathrm{Var}}\left[ {#1} \right]}
\newcommand{\ip}[2]{\left( {#1}, {#2} \right)}
\newcommand{\U}{\mathcal{U}}
\newcommand{\normim}[1]{\left\| {#1} \right\|_{\scriptscriptstyle L^{2}(\Omega^{*})}}
\newcommand{\avemu}[1]{\mathrm{E}\left({#1}\right)}
\newcommand{\ave}[1]{\left\langle {#1} \right\rangle}
\newcommand{\prob}[1]{\mathrm{Prob}\left\{ {#1} \right\}}
\newcommand{\ind}[1]{\mathrm{\chi}_{\scriptscriptstyle {#1} }}
\newcommand{\NISP}{\mathcal{S}}
\newcommand{\ipmu}[2]{\left( {#1}, {#2} \right)_\mu}
\newcommand{\norm}[1]{\left\| {#1} \right\|_{\scriptscriptstyle L^{2}(\Omega)}}
\newcommand{\pard}[2]{\frac{\partial{#1}}{\partial{#2}}}
\def\xbold{{\vec{x}}}
\renewcommand{\vec}[1]{{\mathchoice
                     {\mbox{\boldmath$\displaystyle{#1}$}}
                     {\mbox{\boldmath$\textstyle{#1}$}}
                     {\mbox{\boldmath$\scriptstyle{#1}$}}
                     {\mbox{\boldmath$\scriptscriptstyle{#1}$}}}}

\newcommand{\pd}{\pard}
\newcommand{\geoclaw}{{\sc GeoClaw}\xspace}
\newcommand{\tohoku}{T\={o}hoku\xspace}
\newcommand{\dx}{\ensuremath{\Delta x}}   % Delta x
\newcommand{\dy}{\ensuremath{\Delta y}}   % Delta y
\newcommand{\dt}{\ensuremath{\Delta t}}   % Delta t
        
\journal{Ocean Modelling}

\begin{document}

\ifpdf
\DeclareGraphicsExtensions{.pdf, .png, .jpg, .tif}
\else
\DeclareGraphicsExtensions{.png, .jpg, .tif, .eps}
\fi

\begin{frontmatter}

\title{Estimation of Manning's Friction Coefficient using DART Buoys Data during
\tohoku Tsunami}
\date{\today}

\author[kaust]{Ihab Sraj}
\ead{ihab.sraj@kaust.edu.sa}
\ead{Department of Physical Sciences and Engineering, King Abdullah University for Science and Technology, Thuwal, Saudi Arabia}
\author[ut]{Kyle T. Mandli}
% \ead{kyle@ices.utexas.edu}
% \ead[http://users.ices.utexas.edu/~kyle]{http://users.ices.utexas.edu/~kyle}
\author[duke]{Omar M. Knio}
\author[ut]{Clint N. Dawson}
\author[kaust]{Ibrahim Hoteit}

\address[kaust]{Department of Physical Sciences and Engineering, King Abdullah University for Science and Technology, Thuwal, Saudi Arabia}
\address[ut]{Institute for Computational Engineering and Science, University of Texas at Austin, 201 E 24th ST. Stop C0200, Austin, TX 78712-1229, USA}
\address[duke]{Department of Mechanical Engineering and Materials Science, Duke University, 144
Hudson Hall, Durham, North Carolina 27708, USA}
\date{\today}

\begin{abstract}

Tsunamis have been lately catastrophic due to the warnings
issued late and to the inaccurate predictions of water elevations. 
Tsunami computational models are often employed to explore multiple 
different scenarios to predict water elevations due to the abundance 
of uncertainty in the input data. However, accurate estimation of water elevations 
requires accurate estimation of many model parameters,
in particular the Manning's $n$ friction coefficient that plays an important role in the 
tsunami governing equations. Our objective here is to present an efficient method for the 
estimation of Manning's $n$ coefficient characterized by three different parameters 
assumed constant in three regions: $n_1$ on-shore, $n_2$ near-shore and $n_3$ deep-water. 
This method uses Polynomial Chaos (PC) to build an inexpensive surrogate for the 
\geoclaw model and then employs the Bayesian inference that uses the DART buoys data
obtained during \tohoku tsunami to estimate the three uncertain parameters. The efficiency of our approach stems from 
using a surrogate model that reduces the computational burden of the Markov Chain Monte-Carlo 
sampling needed in the Bayesian inference. The PC surrogate is also used to perform a sensitivity analysis. 
Our results indicate that Manning's $n$ coefficients have a Maximum-A-Posteriori (MAP) values of $n_2=0.011$ and $n_3=0.180$, while for $n_1$ no meaningful MAP value can be determined using the available data. 


\end{abstract}

\begin{keyword}
tsunami \sep Manning's $n$ friction coefficient \sep sensitivity analysis \sep polynomial chaos \sep Bayesian inference
\end{keyword}

\end{frontmatter}
\linenumbers

% \renewcommand{\thefootnote}{\fnsymbol{footnote}}

\begin{center}
\begin{Large}
{\bf Polynomial Chaos for the Estimation of Manning's Friction}\\

\end{Large}
\bigskip
\bigskip
Ihab Sraj$^1$, Kyle Mandli$^2$ \& Ibrahim Hoteit$^1$\\
\bigskip
$^1$Department of Physical Sciences and Engineering,
King Abdullah University for Science and Technology, Thuwal, Saudi Arabia \\
$^2$Department of Engineering, Austin, TX 
\end{center}

\vspace{5.5cm}

\begin{tabbing}
Corresponding Author: \hspace{5mm} \=  \\
       \> Department of Physical Sciences and Engineering \\
       \> King Abdullah University for Science and Technology \\
       \> Thuwal, Saudi Arabia\\
       \> \\
Phone: \>  \\
Fax:   \>  \\
Email: \>    \\
\\
Submitted to: \> {\it TBD} \\
\> \today \\
\\
Revised: \>  

\end{tabbing}





% \clearpage

%!TEX root = paper.tex

\section{Introduction} \label{sec:intro}

Tsunamis in the past decade have been responsible for some of the most  deadly
and costly natural disasters ever recorded. Coastal communities  have faced this
hazard by assessing the risk they pose, attempting  to make informed decisions
about the likelihood that such an event would  occur with cost to life and
property.  Toward this goal, computational models of  tsunami events are often
employed to explore multiple different scenarios to predict the water elevation
caused by the tsunamis. The accurate prediction  of water elevations, however,
requires accurate estimation of many model parameters that are either measured
directly, defined empirically, or estimated from a collection of observed data.
Unfortunately, since tsunami events are relatively rare,  there is an abundance
of uncertainty in the input data for these computational  models ranging from
effects of the domain, such as bathymetry and friction parameterizations,  to
the earthquake source.  In this study, we aim to estimate the Manning's $n$
friction coefficient, a parameterization of the effect of bottom friction 
commonly used in tsunami models.  We aim at quantifying the uncertainty in the 
predicted water elevation  and employ
a Bayesian inverse modeling approach to estimate the Manning's $n$ coefficient
using observations of water elevation measured during a tsunami event.

Previous work looking into uncertainty in the input for tsunami simulations
focused on the earthquake source but primarily as a way of
constraining the various proposed earthquake scenarios only after the event (see
for example \cite{MacInnes:2013cr}).  A similar approach for landslide-generated
tsunami was presented in Sarri \emph{et al.}~\cite{Sarri2012}, by building a statistical surrogate
using an emulator. The implemented emulator was based on a Gaussian
process that requires using a combination of prior knowledge about the simulator
and appropriate choices of functions and parameters. Their work was presented as
a proof-of-concept case study where they only performed basic statistical and
sensitivity analysis.  Changes to the friction parameterization has been shown in 
the past to lead to significant changes to inundation levels \cite{Myers:2001el, 
Jakeman:2010hk}, but these previous studies were limited in the range of the 
values and locations used in the parameterization.  
In a recent work, Mayo \emph{et al.}~\cite{Mayo:2014} reformulated a statistical data assimilation method
generally used in the estimation of model states to estimate the Manning's $n$
coefficient. They used a low-dimensional representation of
Manning's $n$ coefficients and recovered it by assimilating synthetic water elevation data
in a storm surge setting.

In this work, we focus our attention on the uncertainty in the friction
parameterization used in tsunami simulations. We present a \alert{three}-dimensional 
representation of Manning's $n$ coefficient and an efficient method
to estimate this representation using real water elevation data. The method proposed here 
is based on recent developments in Uncertainty Quantification (UQ) methods that allow 
probing of the sensitivity of realistic tsunami  models to uncertain parameters
without modifying the forward model and inferring those parameters
from a number of observations.  In particular, we implement an inverse modeling
approach that relies on the Bayesian inference technique.  We resort to
Polynomial Chaos (PC) expansions to construct a faithful surrogate model of the
response of the large and complex geophysical model simulations \cite{Najm:2009,Alexanderian2012,sraj:2013a,sraj:2013b}.
The construction of the surrogate enables us to efficiently implement the Bayesian
inference technique as it dramatically reduces the cost of sampling the
posterior distribution.

PC expansions have been developed and applied in the engineering community to
quantify uncertainties in numerical simulations.  The main advantage of
using PC expansions is the ability of propagating input uncertainties through 
large, complex and non-linear models leading to the corresponding output uncertainties. 
Several studies already investigated the efficiency of PC expansions for 
UQ in oceanic simulations~\citep{thacker2012,ashwanth2010,Alexanderian2012}. 
Recently, PC was combined with Bayesian inference to estimate the wind drag coefficient 
using temperature data collected during Typhoon Fanapi 2010. The same problem was 
solved using two different techniques: the adjoint method and Markov Chain Monte Carlo
(MCMC)\cite{sraj:2013a,sraj:2013b}.

In this work, we follow a similar procedure as in \cite{sraj:2013a,sraj:2013b}
to estimate Manning's $n$ coefficient using the \tohoku tsunami
event as a case-study. The quantity of interest is taken to be the surface
elevation $\eta$ and is compared to the available NOAA Deep-ocean Assessment and
Reporting Tsunamis (DART) buoys.  The forward model runs are performed using the
\geoclaw package following the setup in MacInnes \emph{et al.}~\cite{MacInnes:2013cr} 
including the two source models analyzed there.

The structure of this manuscript is as follows. First, we discuss the uncertainties 
of the Manning'g $n$ coefficient to motivate the UQ problem in Section~\ref{sec:manning}
and propose a \alert{three}-dimensional representation of $n$. 
Section~\ref{sec:tohoku} describes our case-study, the \tohoku tsunami,
summarize the forward simulations setup, and present the DART buoys observational data. 
Section~\ref{sec:formu} presents the two main methods employed in the UQ analysis:
Bayesian inference that brings together observations 
and model results (Section~\ref{sec:inference}) 
and PC expansions that are used to build the surrogate model (Section~\ref{sec:uqpce}). Section~\ref{sec:results} describes the main results in 
three subsections: first, we perform error and convergence studies
as evidences that the PC surrogate is a faithful 
representation of the \geoclaw model; second, we use the PC surrogate to 
perform a statistical analysis and compute the  sensitivity of simulated water elevations
to the uncertain input parameters; and third, we present the results of the
posterior distributions obtained using Bayesian inference by MCMC. A discussion of our findings and methodology is finally presented in Section~\ref{sec:conc}.

\comment{
\begin{itemize}
    \item There are two places where we refer to the parameterization as three-dimensional, I am not sure if this is really three-dimensional.  I know that we are using depth contours but the parameterization is two-dimensional.  I have a feeling that readers would take issue with this.
    \item Ihab: the representation is 3D meaning we used three different values of n in three regions. The same concept was presented in Mayo paper where they did 2D representation.
    \item I think the order here needs to be revised to talk about the forward model and the equations we are solving first since the Manning's $n$ discussion refers to equations there. 
    \item Ihab: Our main objective is to show the UQ techniques and how they work. The equations are known and we do not need to stress on them. We can even remove the whole equation or forward model section.
\end{itemize}
}


%!TEX root = paper.tex

\section{Approach}

The approach taken to quantify the surface displacement uncertainties associated with the Manning's $n$ friction parameterization uses a polynomial chaos approach coupled with a shallow water forward model as applied to a real event, the \tohoku tsunami of 2011.  Here we briefly described the setup of the forward model and the prior of the model used.

\subsection{Forward Model}

The forward model we use is the \geoclaw package, a finite volume wave-propagation class that solves the two-dimensional nonlinear shallow water equations  
\begin{equation} \label{eq:swe}
    \begin{aligned}
    &\pd{}{t} h + \pd{}{x} (hu) + \pd{}{y} (hv) = 0, \\
    &\pd{}{t}(hu) + \pd{}{x} \left(hu^2 + \frac{1}{2} g h^2 \right ) + \pd{}{y} (huv) = ~~ fhv - gh \pd{}{x} b - C_f |\vec{u}| u \\
    &\pd{}{t} (hv) + \pd{}{x} (huv) + \pd{}{y} \left (hv^2 + \frac{1}{2} gh^2 \right) = -fhu - gh \pd{}{y} b - C_f |\vec{u}| v
    \end{aligned}
\end{equation}
where $h$ is the depth of the water column, $u$ and $v$ the velocities in the longitudinal and latitudinal directions respectively, $g$ the acceleration due to gravity, $b$ the bathymetry, $f$ the Coriolis parameter, and $C_f$ the bottom friction coefficient.  As is common, $C_f$ is calculated given a Manning's $n$ parameterization such that
\begin{equation}
    C_f = \frac{g n^2}{h^{5/3}}
\end{equation}
where $n$ is determined from empirical roughness studies.

The input data 


\subsubsection{Source Term Evaluation} \label{ssub:source_term_numerics}

Since the effect of friction on the water column is the primary focus of this study, we give a brief description of the process by which friction and the other source terms are included in \geoclaw.  We can write hyperbolic balance laws such as \eqref{eq:swe} commonly in the form
\begin{equation}
    q_t + f(q)_x + g(q)_y = \Psi(q)
\end{equation}
where $q$ are the conserved quantities, $f(q)$ and $g(q)$ the fluxes in the appropriate directions, and $\Psi(q)$ the source terms.  In the case of \eqref{eq:swe} these can be written
\begin{align}
    q = \begin{bmatrix}
        h \\ hu \\ hv
    \end{bmatrix} & &
    f(q) = \begin{bmatrix}
        hu \\ hu^2 + \frac{1}{2} gh^2 \\ huv
    \end{bmatrix} & &
    g(q) = \begin{bmatrix}
        hv \\ huv \\ hv^2 + \frac{1}{2} gh^2
    \end{bmatrix}
    \Psi(q) = \begin{bmatrix}
        0 \\ fhv - ghb_x - C_f |\vec{u}| u \\ -fhu - ghb_y - C_f |\vec{u}| v
    \end{bmatrix}.
\end{align}

In \geoclaw, the bathymetry source term is handled directly in the Riemann solver step due to well-balancing concerns.  The Coriolis and friction terms are handled via a operator-splitting approach that solves two simpler problems, $q_t + f(q)_x + g(q)_y = 0$ and $q_t = \Psi(q)$, combining the updates to $q$.  The simplest splitting is Godunov splitting in which each system is solved alternately using the full time step $\dt$.  Although this approach is only first-order accurate the observed splitting errors do no dominate the overall error in practice (see \cite{LevequeBook2002} for a more thorough discussion).

Starting with the friction and Coriolis, the bottom friction terms can be evaluated as
\[
    (hu)_t = -C_f hu ~~~~\text{and}~~~~ (hv)_t = -C_f hv
\]
and are solved using a backwards Euler method for computing the loss of momentum so that
\[
    (hu)^{n+1}_{ij} = \frac{(hu)^n_{ij}}{1 + (C_f)_{ij} \dt} ~~~~\text{and}~~~~
    (hv)^{n+1}_{ij} = \frac{(hv)^n_{ij}}{1 + (C_f)_{ij} \dt}
\]
where the drag coefficient is computed using the previous time step's state as
\[
    (C_f)_{ij} = \frac{gn_{ij}^2}{(h^n_{ij})^{-7/3}} \sqrt{[(hu)^n_{ij}]^2 + [(hv)^n_{ij}]^2} \left[1-\left(\frac{h_{\text{break}}}{h^n_{ij}}\right)^{\theta_f} \right]^{\gamma_f / \theta_f}.
\]
The values $h_{\text{break}}$, $\theta_f$, and $\gamma_f$ are constant parameters that define the hybridization between the Chezy and Manning's $n$ formulation from \eqref{eq:friction_term}.  The Coriolis terms
\[
    (hu)_t = -f hv ~~~~\text{and}~~~~ (hv)_t = f hu
\]
are evaluated using a matrix exponential up to the 4th term in the series such that the update becomes
\[
    \begin{bmatrix}
        hu \\ hv     
    \end{bmatrix}^{n+1}_{ij} = 
    \begin{bmatrix}
        hu \\ hv     
    \end{bmatrix}^{n}_{ij} \cdot 
    \begin{bmatrix}
        1 - \frac{1}{2} (f \dt)^2 + \frac{1}{24} (f \dt)^4 & f \dt - \frac{1}{6} (f \dt)^3 \\
        - f \dt + \frac{1}{6} (f \dt)^3 & 1 - \frac{1}{2} (f \dt)^2 + \frac{1}{24} (f \dt)^4
    \end{bmatrix}
\]
where $f = 2 \Omega \sin y$ with $\Omega = 2 \pi / 8.61642\times10^4$ and $y$ the longitudinal coordinate in radians. 


\begin{enumerate}
    \item Finite volume wave-propagation approach
    \item Adaptive mesh refinement
    \item Riemann solver - included inundation and well-balanced properties
    \item Implementation of friction as a source-term splitting
\end{enumerate}


\geoclaw is Based on work by \cite{MacInnes:2013cr}, \cite{Berger:2011du}, \cite{Berger:2011vi}, \cite{George:2008aa}

\subsection{\tohoku Tsunami}

The \tohoku earthquake and tsunami of 2011 has been the subject of a number of studies due to the wealth of observational evidence and severity of the tsunami.  The earthquake itself had an estimated magnitude of $\text{M}_\text{w}$ 9.0 and caused massive devastation across Japan.  In this study we have based our simulation on a recent study that attempted to ascertain which source earthquake model matched the observed data optimally \cite{MacInnes:2013cr}.  

\begin{enumerate}
    \item Note that we do not have the bathymetry used in this paper
    \item Figure out which source we are using
\end{enumerate}

\subsection{Observations}

The observations used in this study are from the Deep-ocean Assessment and Reporting of Tsunamis (DART) buoy system developed and maintained by the National Oceanic and Atmospheric Administration (NOAA).  The purpose of the network is to provide early-warning detection and forecasting of tsunami propagation in the Pacific Ocean \cite{Percival:2011}.  The DART buoys closest to the earthquake source of the \tohoku tsunami were buoys 21418, 21413, 21401, and 21419 whose locations are shown in Figure~\ref{fig:setup} and whose de-tided data for the event are show in Figure~\ref{fig:obs}.

% \begin{figure}[tb]    
%     \centering
%     \begin{subfigure}[b]{0.48\textwidth}
%         \includegraphics[width=\textwidth]{./figures/DART_locations}
%         \caption{}
%         \label{fig:dart_locations}
%     \end{subfigure}
%     \begin{subfigure}[b]{0.48\textwidth}
%         \includegraphics[width=\textwidth]{./figures/DART_data}
%         \caption{}
%         \label{fig:dart_data}
%     \end{subfigure}
%     \caption{Stuff}
%   \end{figure}

\subsection{Parametric Uncertainty}
To represent the uncertainty in the Manning's $n$ coefficients a uniform distribution was assumed as the prior where $n$ was sampled in the interval $[0.005-0.2]$.  The regions

Uniform distribution is assumed as the prior for the three manning coefficients.
such that UN = [0.005-0.2]. The three regions are shown in Figure~\ref{fig:setup}(right)

\subsection{Observations}
During this tsunami, different gauges were installed to observed the surface elevation
The locations of theses gauges are shown in Figure~\ref{setup} .
The gauges are used in the inverse problem to infer the manning coefficients.


\section{Formulation}
\label{sec:formu}
We begin by introducing the methods employed in this work.
The Bayesian approach to inverse problems is described in Section~\ref{sec:inference};
and its application to our problem of inferring the Manning Friction Coefficients is described in Section~\ref{sec:manning}. Finally in Section~\ref{sec:uqpce}, we introduce
a stochastic spectral method for forward uncertainty propagation and accelerating the Bayesian inference.
%The two pieces are brought together in~\ref{sec:setup}.

\subsection{Bayesian Inference}
 \label{sec:inference}
Bayesian inference is a statistical approach to inverse problems
that has recently gained much interest in different applications.
The key idea in this approach is to express all forms of uncertainty
in terms of probability. The first step in a Bayesian inference 
is to formulate the forward problem (response surface/model) using 
a suitable likelihood function and product of conditional probability densities. 
We then formulate the prior distribution over the unknown parameters of the model
using our best belief. Lastly, given some observation data, Bayes rule 
is used to obtain a posterior distribution for these unknown
\citep{sivia}. We review this approach briefly below.

Let the observation data (water surface elevation at the gauges) be denoted by \{$S_i = S_i(x_i,y_i,t_i)\}_{i=1}^m$;
their model predicted counterparts by \{$M_i= M_i(x_i,y_i,t_i)\}_{i=1}^m$
($m$ is the number of observations) and $\{N_i\}_{i=1}^3$ the set of 
three Manning Friction coefficients treated as the unknown parameters. 
Bayes' rule yields:
\begin{equation}
 p(\{N_i\}_{i=1}^3| \{S_i\}_{i=1}^m) \propto 
 p(\{S_i\}_{i=1}^m | \{N_i\}_{i=1}^3) \ p(\{N_i\}_{i=1}^3)  
\label{eq:bayes}
\end{equation}
where $p(\{N_i\}_{i=1}^3)$ is the prior of $\{N_i\}_{i=1}^3$, representing the \emph{a priori} knowledge
about the manning coefficient; 
$p(\{S_i\}_{i=1}^m | \{N_i\}_{i=1}^3)$ is the likelihood function representing
the probability of obtaining the data given the set of parameters $\{N_i\}_{i=1}^3$;
and finally $p(\{N_i\}_{i=1}^3| \{ S_i \}_{i=1}^m)$ is the posterior,
representing the probability that $\{N_i\}_{i=1}^3$ is true given the data $( \{ S_i \}_{i=1}^m )$.

\begin{equation}
p(\{N_i\}_{i=1}^3 | \{ S_i \}_{i=1}^m)
\propto 
p(\{S_i\}_{i=1}^m | \{N_i\}_{i=1}^3) \ p(N_1)p(N_2) p(N_3)  
\end{equation}

To formulate the likelihood function, we let 
	 $\epsilon_i = S_i - M_i $
represents the discrepancy between the model and observations 
and assume that the observations are independent
and normally distributed with mean
zero and variance $\sigma^2$, i.e. 
$\epsilon_i \sim N(0,\sigma^2)$. In this 
case the likelihood function can be written as:
\begin{equation} 
p(\{S_i\}_{i=1}^m |  \{N_i\}_{i=1}^3) 
= 
\prod_{i=1}^m  \frac{1}{\sqrt{2 \pi \sigma^2}} 
\exp \left( \frac{-(S_i - M_i)^2}{2 \sigma^2} \right)  	
\label{eq:likelihood}
\end{equation}

The variance $\sigma^2$ is unknown \emph{a priori}; thus we treat it as a hyper-parameter.
While in general $\sigma^2$ depends on the observations, in cases where the error
amplitude is generally small and does not change throughout space and time, one may use
a simplified model and assume single hyper-parameter $\sigma^2$.

The next step is choosing the prior that should be based 
on some \emph{a priori} knowledge about the parameters. In our case, a uniform
prior for the model parameters is assumed:
\begin{equation} 
p(N_i) = \begin{cases}
		\displaystyle \frac{1}{b_i-a_i} &\text{for~} a_i <  N_i \leq b_i ,  \\
		0 &\text{otherwise}  , 
\end{cases}
\end{equation}
where $ [a_i,b_i]$ denote the parameter ranges defined in 
Equations~(\ref{eq:prior1}--\ref{eq:prior3}).
Regarding the variance, the only information we know 
is that $\sigma^2$ is always positive.
We thus assume a Jeffreys prior \citep{sivia}, expressed as:
\begin{equation} 
p(\sigma^2) =  \begin{cases}
		\displaystyle \frac{1}{\sigma^2} &\text{for~} \sigma^2 > 0,  \\
		0 &\text{otherwise}. 
		\end{cases}
\label{eq:var_pr}
\end{equation}
Consequently, Bayes' theorem gives:
\begin{equation} 
p(\{N_i\}_{i=1}^3,\sigma^2 | \{S_i\}_{i=1}^m) 
\propto
\left[ \prod_{i=1}^m  \frac{1}{\sqrt{2 \pi \sigma^2}} 
\exp \left( \frac{-(S_i - M_i)^2}{2 \sigma^2} \right) \right] 
\ p(\sigma^2)p(N_1)p(N_2) p(N_3)
\end{equation}
where $p(\{N_i\}_{i=1}^3,\sigma^2 | \{S_i\}_{i=1}^m)$  is the joint posterior.

Inferring the drag coefficient parameters requires 
sampling the posterior. In general, when the space of the unknown 
parameters is multidimensional, a suitable computational strategy is 
the Markov Chain Monte  Carlo (MCMC) method. 
We rely on an adaptive Metropolis MCMC \citep{Gareth2009,Haario2001} to
sample the posterior distribution accurately and efficiently.
This MCMC phase requires repeated (tens of
thousands of) Geoclaw simulations initialized with different values of the
uncertain parameters; this step is prohibitively expensive. An
alternative is to construct a surrogate model that requires a much
smaller ensemble of GeoClaw runs, and that can be used instead
at a significantly reduced computational cost.  Here, we rely on
PCEs to build the surrogate, which, in addition
also provide statistical properties,  such as the mean, variance and sensitivities,
efficiently.
\subsection{Inferring Manning Friction Coefficient}
 \label{sec:manning}
 
We seek to infer the Manning's n coefficients from water surface elevation
data measured at different gauges as shown in Figure~\ref{fig:setup}.
Bayesian inference can be applied directly to infer the uncertain parameters
using Equation~\eqref{eq:post}. In our case, the observation data $\vec d$ 
are the water surface elevation at the gauges;
their model predicted counterparts $\vec G$ is predicted by \geoclaw.
The uncertain parameters $\vec \theta$ will be denoted by $\{N_i\}_{i=1}^3$ for the
three Manning's coefficients treated as the unknown parameters. 
In this case  Equation~\eqref{eq:post} can be written as:

\begin{equation} 
p(N_1,N_2,N_3,\sigma^2 | \vec d) 
\propto \frac{1}{\sqrt{2 \pi \sigma^2}} 
 \prod_{i=1}^m  
\exp \left\lbrace \frac{-(d_i - G_i)^2}{2 \sigma^2} \right\rbrace
\ p(\sigma^2)p(N_1)p(N_2) p(N_3)
\label{eq:post_coef}
\end{equation}

To complete the definition of the posterior, we need to choose a proper prior that should be based 
on some \emph{a priori} knowledge about the parameters. In our case, a uniform
prior for the model parameters is assumed as indicated:

\begin{equation} 
p(N_i) = \begin{cases}
		\displaystyle \frac{1}{b_i-a_i} &\text{for~} a_i <  N_i \leq b_i ,  \\
		0 &\text{otherwise}  , 
\end{cases}
\end{equation}
where $ [a_i,b_i]$ denote the parameter ranges defined as.
Regarding the variance, the only information we know 
is that $\sigma^2$ is always positive.
We thus assume a Jeffreys prior \citep{sivia}, expressed as:

\begin{equation} 
p(\sigma^2) =  \begin{cases}
		\displaystyle \frac{1}{\sigma^2} &\text{for~} \sigma^2 > 0,  \\
		0 &\text{otherwise}. 
		\end{cases}
\label{eq:var_pr}
\end{equation}

Inferring the coefficients requires 
sampling the posterior. In general, when the space of the unknown 
parameters is multidimensional, a suitable computational strategy is 
the Markov Chain Monte  Carlo (MCMC) method. 
We rely on an adaptive Metropolis MCMC \citep{Gareth2009,Haario2001} to
sample the posterior distribution accurately and efficiently.




\subsection{Accelerating Bayesian Inference}
\label{sec:uqpce}

The four-dimensional posterior in Equation~\eqref{eq:post_coef} can be directly
explored via MCMC; this requires repeated simulations (tens of thousands) of the forward \geoclaw model, 
once for every proposed set of parameters of the Markov chain~\cite{MarzoukNajm2009,Malinverno2002}. While a single \geoclaw simulation
takes $\sim 15~$mins, depending on the details of the MCMC algorithm used, it is desirable 
to avoid running the forward model at every realization of the MCMC. This is achieved by constructing a 
surrogate model that requires a much
smaller ensemble of \geoclaw runs, and that can be used instead
at a significantly reduced computational cost.  Here, we rely on
Polynomial Chaos expansions for accelerating Bayesian inference in this context 
by building a surrogate model, which, in addition can efficiently
provide statistical properties, such as the mean, variance and sensitivities. 

\subsubsection{Polynomial Chaos}

Polynomial Chaos (PC) is a probabilistic methodology that expresses the 
dependencies of model outputs on the uncertain model inputs
as a polynomial truncated expansion. This method has been developed in 
the engineering community to represent uncertainties in the output of 
numerical simulations~\citep{Villegas2012,Lin2009,Xiu2004}
due to the uncertainties in a model's input. We briefly describe the PC
method below; for more details 
the reader is referred to \citep{LeMaitreKnio2010}.

Let $U=U(\bm{x},t,\xxi)$ denote a quantity of 
interest (QoI) that is the output of a computational model.
$U$ is function of space, $\bm{x}$, and time, $t$, and 
also depends on the canonical vector of random variables $\xxi=(\xi_1,...,\xi_n)$
that parameterize the uncertain inputs. 
PC expresses $U$ in the form:

\begin{equation}
  U(\xbold,t,\xxi) \doteq \sum_{k = 0}^P U_k(\xbold,t) \Psi_k(\xxi),
\label{eq:stochseries}
\end{equation} 
where $U_k(\xbold,t)$ are the polynomial coefficients, and
$\Psi_k(\xxi)$ are functions that form an orthogonal basis of an underlying probability
space. The total number of terms in the PC expansion is
$P+1 = \frac{(d+p)! }{n!\ p!}$ where $n$ is the number of stochastic dimensions and $p$ is the highest order
polynomial order. 

The choice of the basis is dictated by the probability density
function of the stochastic variable $\xxi$, which appears as a weight
function in the probability space's inner product:

\begin{equation}
 \left<\Psi_i,\Psi_j\right> = \int \Psi_i(\xxi) \;\Psi_j(\xxi) \; \rho(\xxi) \; \mbox{d}\xxi=\delta_{ij}\ave{\Psi_i^2},
\label{eq:inner}
\end{equation}
where $\delta_{ij}$ is the Kronecker delta.
For uniform
distributions, the basis functions are scaled Legendre polynomials.
For multi-dimensional problems the basis functions are
tensor products of 1D basis functions~\cite{LeMaitreKnio2010}.

The series representation ~\eqref{eq:stochseries} can be viewed as a spectral expansion
of $U$ along the stochastic dimensions. It can also be seen as
combination of approximation and probabilistic frameworks; this
 has proven extremely useful in solving UQ problems~\cite{Xiu:2003,Lin2009}. The existence and convergence of this series is asserted by the Cameron-Martin theorem \citep{Cameron:1947} with the condition of $U$ having a finite variance.
The series rate of convergence, and hence the number of terms to retain, depends on the smoothness of
$U$ with respect to $\xxi$. The series converges spectrally fast with $P$
when $U$ is smooth; the convergence rate becomes algebraic
when $U$ has finite smoothness \citep{Canuto:2006}. In practice the series convergence is monitored 
via various error metrics as discussed in the results section.

\subsubsection{Non Intrusive Spectral Projection (NISP)}
The computation of the coefficients of the PC expansions $U_k$
can be done using a number of procedures. Here we adopt a non-intrusive
approach that allows the use of the forward model \geoclaw as a black box
with no code modifications required. PC expansion coefficients are determined
based on a set of response \geoclaw simulations at specified set of the uncertain parameters. 
Specifically, we rely on the Non-Intrusive Spectral Projection (NISP) method that exploits the orthogonality of the basis and applies the Galerkin projection to find the PC expansion coefficients as follows:

\begin{equation}
 U_k(\bm{x},t) = \frac{\left< U, \Psi_k \right>}{\left< \Psi_k, \Psi_k \right>} = 
 \frac{1}{\left< \Psi_k, \Psi_k \right>} 
 \int U(\bm{x},t,\xxi) \Psi_k(\xxi) \rho(\xxi) \mbox{ d}\xxi.
\end{equation}
This orthogonal projection minimizes the $L_2$ error on the space spanned by the basis.
Using NISP the stochastic integrals are solved using a numerical quadrature to obtain:

\begin{equation}
  \left< U, \Psi_k \right> 
\approx \left< U, \Psi_k \right>_Q
= \sum_{q=1}^Q U(\xxi_q) \Psi_k(\xxi_q) \omega_q,
\end{equation}
where the subscript $Q$ refers to approximating the inner product integral with
quadrature, and $\xxi_q$ and $\omega_q$ are multi-dimensional quadrature points and weights,
respectively. The quadrature order should be commensurate with the
truncation order, and should be high enough to avoid aliasing artifacts.
The choice of quadrature rule is hence critical to the performance
of the PC (in its NISP version at least). \comment{In the current work, we employ the 
tensorized Gaussian quadrature that yields 125 quadrature points for the set of three input parameters.
The computational cost of this ensemble is low compared to the cost of the 
tens of thousands of runs using the forward model.}

The computation of the ${U}_k$ can thus be expressed as a matrix-vector product of the form:

\begin{equation} 
 U_k(\bm{x},t)=\sum_q \Pi_{kq} U(\bm{x},t,\xxi_q),\;\;\;
 \Pi_{kq}=\frac{\Psi_k(\xxi_q)\omega_q}{\left< \Psi_k, \Psi_k \right>},
\end{equation} 
where $\Pi_{kq}$ is the projection matrix and $U(\bm{x},t,\xxi_q)$ is obtained
from an ensemble of the deterministic model realizations with the uncertain parameters set at
the quadrature value $\xxi_q$. 


\subsubsection{Statistical moments and sensitivity analysis}
The identification of the inner product weight function
with the probability distribution of $\xxi$ simplifies the calculations of statistical moments of $U$. 
Noting that since $\Psi_0(\xxi)$ is a constant that is normalized so that 
$\left<\Psi_0,\Psi_0\right>=1$, the expectation and variance of $U$ can be computed as:

\begin{equation}
 E[U] = \int U \, \rho(\xxi) \, \mbox{d}\xxi=\left< U,\Psi_0\right> = U_0,  
 \label{eq:mean}
\end{equation}
and \begin{equation}
 E[(U-E[U])^2] = \int (U-E[U])^2 \, \rho(\xxi) \, \mbox{d}\xxi=\sum_{k=1}^P U_k^2
 \label{eq:sigma}
\left<\Psi_k,\Psi_k\right>.
\end{equation}

PC representations also enable conducting
efficient global sensitivity analysis that quantify the
contribution of different random input parameters to the variance in the output.
This can be done by computing the so-called {\it total} 
sensitivity index $T_i$ that measures the contribution of
the $i^{th}$ random input to total model variability by
computing the fraction of the total variance due to all the terms in the
PC expansion that involve $\xi_i$~\citep{LeMaitreKnio2010,Crestaux,Sudret}
as follows:

% of random variables from the PC representations or Sobol decomposition~\citep{Sobol:1993,Homma:1996,Sobol:2001}. The total sensitivity index  
%
%To get the total sensitivity corresponding to the uncertain
%input $\xi_i$ we compute the total index:

\begin{equation} \label{eq:T-hard}
   T_i =
         \frac{\displaystyle
               \sum_{k \in K_i} U_k^2 \ave{\Psi_k^2}}
              {\displaystyle\sum_{k = 1}^P U_k^2 \ave{\Psi_k^2}},
\end{equation}
where \[
   K_i = \left\{ k \in \{1, \ldots, P\} :
           \vec{\alpha}^k_i > 0 \right\}
        \]
        and $\vec{\alpha}^k$ is the multi-index associated with $k^{th}$ term in the
PC expansion~\cite{LeMaitreKnio2010}.

%Using Equation~\eqref{eq:T-hard}, the computation of $T_i$ is straightforward.





\section{Results}
\label{sec:results}
We begin the results section by performing
an error and convergence analysis of the
constructed surrogate using the PC expansions
to establish its validity in Section~\ref{sec:analysis}.
Then in Section~\ref{sec:forward}, we present
statistical analysis to quantify the uncertainty in the predicted surface elevation
as well as a sensitivity analysis to rank the impact of the different Manning's 
roughness coefficients on this uncertainty. 
Finally, in Section~\ref{sec:inverse}, we present the results of the inverse
problem where we determined the posterior distributions of the input parameters 
and analyzed them in light of the available gauge data.
\subsection{Error and convergence study}
\label{sec:analysis}

Figure~\ref{fig:rlzs} plots the evolution of the
water surface elevation predicted at the four different gauges 
using \geoclaw for the 125 different realizations 
required to compute the PC expansions. We notice that the 
variability in water surface elevation is insignificant in the first 
hour at all gauges as the plots of the different realizations superimpose.
Later in time, the variability starts to increase at the different gauges 
as indicated from the thickness of the bands formed by the plots of the realizations.
This variability appears to be significant at $t\sim$ 5400~s, 6000~s, 3600~s and 7200~s
for gauges 21401, 21413, 21418, 21419, respectively.
This is consistent with the distance from the gauge to the source of the earthquake
(epicenter located approximately 72 km east of \tohoku) where gauge 21418 is the closest and gauge 21419 is the farthest as shown in Figure~\ref{fig:setup_buoy_locations}.
The uncertainty in the prediction of water surface  elevation persists till the end of the simulations
at all gauges.

In order to check the consistency of the PC approximation, we compare
water surface elevation from the realizations 
with those obtained from the PC surrogate. The different curves (not shown) 
reveal an excellent agreement for all times and gauges locations.
To quantify this agreement, we define
an error metric that measures the relative normalized root mean-square error between the left hand side function 
in Equation~(\ref{eq:stochseries}) and its PC representation at the sampling points:

\begin{equation} 
   E = \frac{\displaystyle
         \left(\sum_{\xxi \in \NISP} \left|U(\xxi) - \sum_{k = 0}^{P}
U_k\Psi_k(\xxi)\right|^2
         \right)^{1/2}}
        {\displaystyle
          \left(\sum_{\xxi \in \NISP} \left|U(\xxi)\right|^2\right)^{1/2} 
          },
\label{eq:error}
\end{equation}
where $\NISP$ is the 125-member ensemble obtained to construct the PC surrogate. 
This error metric calculated at the different gauge locations is shown in Figure~\ref{fig:error};
the largest relative normalized error for 
water surface elevation is about 0.1\%. 


%Contour maps of the relative normalized error for the entire simulation region are 
%shown in Figure~\ref{fig:error2D} for various depths and dates to confirm the error trends of the
%analysis box.  The error is largest after the wind
%intensifies (bottom row) for all depths. For either day, the maximum
%error is located at 50~m which coincides with the depth of the
%original mixed layer measured by the AXBT. The maximum magnitude
%recorded is about 1\% and occurs on Sep~18. The
%elevated error region is located to the right of the storm and
%extends from the surface down to 50~m, after which the impact of the
%input uncertainty decreases substantially.  For the purpose of our
%current study, and given that the majority of AXBT data are at
%depth, the surrogate's errors are considered acceptable and small.

 
A final check for the validity of the PC approximation 
consists of verifying whether the probability density
functions (pdfs) of water surface elevation at the different gauge locations
converges with increased order of the PC expansion  \comment{add referece}.  Sample
water surface elevation pdfs are shown in Figure~\ref{fig:pdfs2}
and Figure~\ref{fig:pdfs3} where
the different curves correspond to increased order of PC ($p= 1-5$).
The plots indicate double peaked distributions that are
well-resolved with PC order $p=3$ at t = 7200 s as shown in Figure~\ref{fig:pdfs2}.
However, at t = 10800 s further refinement is needed as the pdfs are
sensitive to the refinement up to order $p=4$ but then becomes weakly 
insensitive with additional refinement  ($p = 5$) as shown in Figure~\ref{fig:pdfs3}. 
We, therefore, used PC order $p = 5$ in all computations below.
This test and the various error metrics presented above provide confidence that the PC expansion is a faithful 
model surrogate that can be used in both the forward and inverse problems. 


\subsection{Statistical and sensitivity analysis}
\label{sec:forward}
We now exploit the PC surrogate to study the statistics of the 
water surface elevation, and to quantify its sensitivity to the
uncertain input parameters and thus to anticipate their impact on
the inverse problem.  In doing so, it is emphasized that 
no additional Geoclaw simulations were needed to obtain
the information presented below, rather it is obtained either directly 
from the coefficients of the PC surrogate or by sampling the corresponding
representations for different values of $\xxi$.

Figure~\ref{fig:ave} shows the evolution of
PC mean water surface elevation and its two standard deviations
bounds at the four gauge locations as indicated in each panel.  
The standard deviation is insignificant during the two hours
of the simulations and therefore was not included in the plots.
This standard deviation however increases as the tsunami increase
and its effect approaches the gauge locations. 
To quantity the contribution of each
uncertain parameter to the variance in water surface elevation, we calculate the total sensitivity index 
using the PC coefficients ~\citep{Alexanderian2012,Sudret,Crestaux}. The total sensitivity index
of  each of the uncertain parameters is shown in
Figure~\ref{fig:sens} for the four gauges. The Manning's roughness coefficient
at the shore $N_2$ is clearly dominant and contributes
most to the  water surface elevation variance compared to the other two 
Manning's roughness coefficients;
this is true throughout the simulation time.  The Manning's roughness coefficient
in the ocean $N_{3}$ at gauge number 21419 exhibits small sensitivity index 
the during the second hour of simulation and Manning's roughness coefficient
in the land $N_1$ appears to be an insignificant contributor
to the variance.
\begin{figure}[h]
\begin{tabular}{clc}
        
\includegraphics[width=0.475\textwidth]{../figures/musigma1.pdf} &
\includegraphics[width=0.475\textwidth]{../figures/musigma2.pdf} \\
\includegraphics[width=0.475\textwidth]{../figures/musigma3.pdf} &
\includegraphics[width=0.475\textwidth]{../figures/musigma4.pdf}
\end{tabular}
\caption{Evolution of PC mean water surface elevation at different gauge locations.}
\label{fig:ave}
\end{figure}
\begin{figure}[h]
\begin{tabular}{clc}
\includegraphics[width=0.475\textwidth]{../figures/sens1.pdf} &
\includegraphics[width=0.475\textwidth]{../figures/sens2.pdf} \\
\includegraphics[width=0.475\textwidth]{../figures/sens3.pdf} &
\includegraphics[width=0.475\textwidth]{../figures/sens4.pdf}
\end{tabular}
\caption{Total sensitivity index of different input parameters.}
\label{fig:sens}
\end{figure}


%The impact of $\alpha$ and $V_{max}$ for fixed $m=0$ is illustrated in Figure \ref{fig:response}
%for the area-averaged SST within the $42~km\times42~km$ analysis region; the
%different panels illustrate the time evolution of the SST response
%surfaces. Their most striking features are the relatively flat
%horizontal contours on Sep~17 and Sep~18 indicating that SST depends
%only mildly on $V_{\max}$ even during peak winds;
% unsurprisingly these contours turn completely flat on Sep
%19 (and afterwards) when the winds dip below 10~m/s. The multiplicative
%factor $\alpha$, however, exerts a strong influence on SST even
%under mild wind conditions.  This influence is relatively
%weak for the low $\alpha$ range and increases substantially for the
%higher $\alpha$ as evidenced by the packed contours.  The temperature
%response surfaces at 50-m and 200-m depths (not shown) exhibit
%roughly the same structure as the surface, with milder
%dependence on $V_{\max}$ even during the peak winds of Sep~18.

The same statistical analysis can be performed for the
entire domain in 2D. Figure~\ref{fig:mean2d}(top row) shows
the PC mean water surface elevation for the considered computational
domain at three different times as indicated in the title of each panel.
The standard deviation is also shown in Figure~\ref{fig:mean2d}(bottom row).
\begin{figure}[h]
        \begin{tabular}{ccc}
\hspace*{-65pt}
\includegraphics[width=0.45\textwidth]{../figures/mean2d1.pdf} &
\hspace*{-65pt}
\includegraphics[width=0.45\textwidth]{../figures/mean2d3.pdf} &
\hspace*{-65pt}
\includegraphics[width=0.45\textwidth]{../figures/mean2d4.pdf} \\
\hspace*{-65pt}
\includegraphics[width=0.45\textwidth]{../figures/sigma2d1.pdf} &
\hspace*{-65pt}
\includegraphics[width=0.45\textwidth]{../figures/sigma2d3.pdf} &
\hspace*{-65pt}
\includegraphics[width=0.45\textwidth]{../figures/sigma2d4.pdf}
\end{tabular}
\caption{PC mean (top row) and standard deviation (bottom row) of the water surface elevation at different times.}
\label{fig:mean2d}
\end{figure}
      
\begin{figure}[h]
\begin{tabular}{clc}
 \hspace*{-65pt}
\includegraphics[width=0.45\textwidth]{../figures/T12d1.pdf} &
\hspace*{-65pt}
\includegraphics[width=0.45\textwidth]{../figures/T12d3.pdf} &
\hspace*{-65pt}
\includegraphics[width=0.45\textwidth]{../figures/T12d4.pdf} \\
\hspace*{-65pt}
\includegraphics[width=0.45\textwidth]{../figures/T22d1.pdf} &
\hspace*{-65pt}
\includegraphics[width=0.45\textwidth]{../figures/T22d3.pdf} &
\hspace*{-65pt}
\includegraphics[width=0.45\textwidth]{../figures/T22d4.pdf} \\
\hspace*{-65pt}
\includegraphics[width=0.45\textwidth]{../figures/T32d1.pdf} &
\hspace*{-65pt}
\includegraphics[width=0.45\textwidth]{../figures/T32d3.pdf} &
\hspace*{-65pt}
\includegraphics[width=0.45\textwidth]{../figures/T32d4.pdf}
\end{tabular}
\caption{Total sensitivity index for $N_1$ (top row) $N_2$ (center row) and $N_3$ (bottom row)
 at different times as indicated.}
\end{figure}
         
        
  \clearpage   
\subsection{Inverse Problem} 
\label{sec:inverse}

Before we attempt solving the inverse problem of parameter estimation, 
we first check the ability of \geoclaw to simulate water surface elevation
realistically. To this end, we present a comparison between the 
(DART) buoys observations  and their \geoclaw model counterparts
for a reference simulation in which the  the Manning's $n$ coefficients were set to their optimal values $n_1=n_2=n_3=0.025$. Figure~\ref{fig:scatter} 
shows a scatter plot that compares the observed 
water surface elevation at the different gauges locations and the  \geoclaw model counterparts. 
The plots show a good agreement between the simulations and the 
observations at different locations and at different times. 
The difference between the observations and simulations is attributed to the uncertainty in the 
Manning's $n$ coefficients and to the measurement errors.
\alert{An interesting observation is the more scatter in the data collection by gauge 21418 which is the closest to
the epicenter of the earthquake and therefore receiving more impact by the tsunami.}

Bayesian inference is now used to estimate the Manning's 
$n$ coefficients ranges that minimize this scatter between 
observed and modeled water surface elevation, and to estimate the variance of the noise in the measured data.
To this end, an adaptive MCMC method is used to sample 
the posterior distributions \citep{Gareth2009,Haario2001} and consequently 
update the Manning's $n$ coefficients distributions in light of the 
observed data. This sampling requires tens of thousands of 
\geoclaw model runs that are prohibitively expensive as each MCMC 
sample requires an independent \geoclaw realization. Instead,
the surrogate model created using PC expansions provides a computationally
efficient alternative that requires only evaluating the PC expansion
for different values of the germ $\xxi$.

The MCMC algorithm required $50,000$ iterations to guarantee convergence
and to obtain the posteriors of the Manning's $n$ coefficients: 
$n_1$,$n_2$ and $n_3$ as well as for the variance $\sigma^2$. Figure \ref{fig:mcmc} 
shows the sample chains for the input parameters for different iterations of the MCMC algorithms. 
The different panels shows well-mixed chains for all input parameters as well as the variance $\sigma^2$.
The chain of  $n_{1}$ span the entire range of the prior meaning the observations are not informative 
concerning this uncertain input.  On the contrary, the $n_{2}$ chain appears to be concentrated in the 
lower end of the parameter range, with values between 0.005 and 0.1. 
While for $n_{3}$, the chain appears to be concentrated in the 
upper end of the parameter range. Finally, the chain for the variance 
appears to be well mixed with a well defined posterior range between 0.02 and 0.022.


Next, the computed MCMC chains are used to determine the posterior 
distributions using kernel density estimation (KDE)
~\citep{Parzen1962,Silverman1986}.  The resulting posterior pdfs 
of the three Manning's $n$ coefficients $n_1,n_2,n_3$ are shown 
in Figure~\ref{fig:pdfs} in addition to the pdfs of $\sigma^2$. 
Note that the first 10000 iterates, associated with the burn-in period, were discarded.  
As expected from the chains shown in Figure
~\ref{fig:mcmc}, the posterior pdfs of $n_1$ appear to be fairly flat, 
and similar to the uniform prior; an indication that 
the observed data were not useful to refine our prior knowledge for $n_1$.,
In contrast, the posterior pdf of $n_2$ exhibits well-defined peak, 
with a Maximum A Posteriori (MAP) estimate of around 0.011
and an extended tail towards the higher Manning's $n$ coefficient values.
While for $n_3$, we observed a posterior that has a well defined peak
of 0.18 but no clear pdf shape. The posterior distribution of the variance 
appears to be well-defined of Gaussian-like shape. 
The MAP value of $\sigma^2=0.021$ can be used to estimate the water surface elevation standard 
deviation estimated to be 0.145~$m$. 
This value is a reflection of the mismatch between the model and 
observed data.


%   0.180222380841571   0.011347632046398   0.185486684362890   0.021050903421753





\section{Discussion and conclusions}
\label{sec:conc}

The present study aimed at estimating Manning's $n$ friction coefficient 
which is crucial in the accurate prediction of water surface elevation in 
tsunami events modeling. We proposed a three-dimensional representation
of Manning's $n$ friction coefficient that was characterized using iso-baths 
to define three distinct regions in the domain and represented each region
with a single Manning's $n$ coefficient. The three regions were as follows:
an on-shore region (initially above sea-level), a near-shore region (between
sea-level and the 200 meter iso-bath), and the deep-water region (deeper than
200 meters) whose Manning's $n$ friction coefficient were denoted by $n_1$, $n_2$, 
and $n_3$ respectively. The estimation relied on a Bayesian
inference approach that sharpens the initial estimates of the three uncertain parameters
using real observations. In our case, we used the DART buoys system data that provides water 
surface elevation data. We specifically used the data collected at four different gauges 
during \tohoku tsunami. To accelerate the Bayesian inference, we relied on the polynomial 
chaos expansions to produce a faithful and efficient surrogate of the forward model \geoclaw. 
The main results of this work were:
\begin{itemize}

\item We built a surrogate model using PC expansions that required 125 forward model runs of \geoclaw.
The surrogate model was tested, validated and proved to be faithful.

\item We used the PC surrogate to quantify the uncertainties in the predicted water surface elevation 
due to the uncertainties in the Manning's $n$ coefficient. To this end, we estimated the mean and the standard deviation of water surface elevation and compared it with the measured data collected at found gauges.

\item A global sensitivity analysis was performed to quantity the contribution of each uncertain parameter to the variance in water surface elevation.  We found that the Manning’s $n$ coefficient at the shore $n_2$ is dominant and
contributes the most to the variance in the water surface elevation compared
to the other two Manning’s $n$ coefficients $n_1$ and $n_3$ for almost the
entire tsunami event.

\item We used Bayesian inference and identified the following MAP
estimates for the three $n$ parameters using MCMC, namely $n_2=0.011$ and
$n_3=0.180$, while for $n_1$ no meaningful MAP value could be determined from the
available data.

\end{itemize}
 
Finally, we note that the present study focused on formulating and estimating a low-dimensional representation
of the Manning's $n$ coefficient using UQ techniques namely Bayesian inference and PC expansions.  A high-dimnensional representation of the Manning's $n$ coefficient would require a large number of forward runs that is computationally expensive. Alternatively, advanced techniques can be implemented to reduce the dimensionality of the problem 
that is the objective of a future study.

\clearpage

\begin{figure}[h]
\centering
\begin{tabular}{clc}
\includegraphics[width=0.45\textwidth]{./figures/topo.pdf}  &
\includegraphics[width=0.45\textwidth]{./figures/coef.pdf} 
\label{setup}
\end{tabular}
\caption{Topography and gauge locations.}
\label{fig:setup}
\end{figure}
%%%%%%%%%%%%%%%%%%%%%%%%%%%%%%%%%%%%%%%%%%%%%%%%%%%%%%%%%%%%%%%%
\begin{figure}[h]      
\centering
\includegraphics[width=0.6\textwidth]{./figures/obs.pdf}
\caption{Observed data of water surface elevation with time at four different gauges.}
\label{fig:obs}
\end{figure}  
%%%%%%%%%%%%%%%%%%%%%%%%%%%%%%%%%%%%%%%%%%%%%%%%%%%%%%%%%%%%%%%%
\begin{figure}[h]
\centering
\begin{tabular}{clc}        
\includegraphics[width=0.6\textwidth]{./figures/rlzs_gauges.pdf} 
\end{tabular}
\caption{125 Geoclaw realizations at different gauge locations.}
\label{fig:rlzs}
\end{figure}
%%%%%%%%%%%%%%%%%%%%%%%%%%%%%%%%%%%%%%%%%%%%%%%%%%%%%%%%%%%%%%%%

\begin{figure}[h]
\centering
\begin{tabular}{clc}        
\includegraphics[width=0.5\textwidth]{./figures/error_gauge1.pdf} &
\includegraphics[width=0.5\textwidth]{./figures/error_gauge2.pdf} \\
\includegraphics[width=0.5\textwidth]{./figures/error_gauge3.pdf} &
\includegraphics[width=0.5\textwidth]{./figures/error_gauge4.pdf} 

\end{tabular}
\caption{125 Geoclaw realizations at different gauge locations.}
\label{fig:error}
\end{figure}   
%%%%%%%%%%%%%%%%%%%%%%%%%%%%%%%%%%%%%%%%%%%%%%%%%%%%%%%%%%%%%%%%
\begin{figure}[h]
\centering

\begin{tabular}{clcl}
\includegraphics[width=0.5\textwidth]{./figures/pdfs1_2.pdf} &
\includegraphics[width=0.5\textwidth]{./figures/pdfs2_2.pdf} \\
\includegraphics[width=0.5\textwidth]{./figures/pdfs3_2.pdf} &
\includegraphics[width=0.5\textwidth]{./figures/pdfs4_2.pdf}
\end{tabular}
\caption{pdf of water surface elevation at the different gauge locations at t = 7200 s.}
\label{fig:pdfs2}
\end{figure}
%%%%%%%%%%%%%%%%%%%%%%%%%%%%%%%%%%%%%%%%%%%%%%%%%%%%%%%%%%%%%%%%
\begin{figure}[h]
\centering
\begin{tabular}{clc}
        
\includegraphics[width=0.5\textwidth]{./figures/pdfs1_3.pdf} &
\includegraphics[width=0.5\textwidth]{./figures/pdfs2_3.pdf} \\
\includegraphics[width=0.5\textwidth]{./figures/pdfs3_3.pdf} &
\includegraphics[width=0.5\textwidth]{./figures/pdfs4_3.pdf}
\end{tabular}
\caption{pdf of water surface elevation at the different gauge locations at t = 10800 s.}
\label{fig:pdfs3}
\end{figure}
%%%%%%%%%%%%%%%%%%%%%%%%%%%%%%%%%%%%%%%%%%%%%%%%%%%%%%%%%%%%%%%%
\begin{figure}[h]
\begin{tabular}{clc}
        
\includegraphics[width=0.475\textwidth]{./figures/musigma1.pdf} &
\includegraphics[width=0.475\textwidth]{./figures/musigma2.pdf} \\
\includegraphics[width=0.475\textwidth]{./figures/musigma3.pdf} &
\includegraphics[width=0.475\textwidth]{./figures/musigma4.pdf}
\end{tabular}
\caption{Evolution of PC mean water surface elevation at different gauge locations.}
\label{fig:ave}
\end{figure}
%%%%%%%%%%%%%%%%%%%%%%%%%%%%%%%%%%%%%%%%%%%%%%%%%%%%%%%%%%%%%%%%

\begin{figure}[h]
\begin{tabular}{clc}
\includegraphics[width=0.475\textwidth]{./figures/sens1.pdf} &
\includegraphics[width=0.475\textwidth]{./figures/sens2.pdf} \\
\includegraphics[width=0.475\textwidth]{./figures/sens3.pdf} &
\includegraphics[width=0.475\textwidth]{./figures/sens4.pdf}
\end{tabular}
\caption{Total sensitivity index of different input parameters.}
\label{fig:sens}
\end{figure}

%%%%%%%%%%%%%%%%%%%%%%%%%%%%%%%%%%%%%%%%%%%%%%%%%%%%%%%%%%%%%%%%

\begin{figure}[h]
        \begin{tabular}{ccc}
\hspace*{-65pt}
\includegraphics[width=0.45\textwidth]{./figures/mean2d1.pdf} &
\hspace*{-65pt}
\includegraphics[width=0.45\textwidth]{./figures/mean2d3.pdf} &
\hspace*{-65pt}
\includegraphics[width=0.45\textwidth]{./figures/mean2d4.pdf} \\
\hspace*{-65pt}
\includegraphics[width=0.45\textwidth]{./figures/sigma2d1.pdf} &
\hspace*{-65pt}
\includegraphics[width=0.45\textwidth]{./figures/sigma2d3.pdf} &
\hspace*{-65pt}
\includegraphics[width=0.45\textwidth]{./figures/sigma2d4.pdf}
\end{tabular}
\caption{PC mean (top row) and standard deviation (bottom row) of the water surface elevation at different times.}
\label{fig:mean2d}
\end{figure}
      %%%%%%%%%%%%%%%%%%%%%%%%%%%%%%%%%%%%%%%%%%%%%%%%%%%%%%%%%%%%%%%%

\begin{figure}[h]
\begin{tabular}{clc}
 \hspace*{-65pt}
\includegraphics[width=0.45\textwidth]{./figures/T12d1.pdf} &
\hspace*{-65pt}
\includegraphics[width=0.45\textwidth]{./figures/T12d3.pdf} &
\hspace*{-65pt}
\includegraphics[width=0.45\textwidth]{./figures/T12d4.pdf} \\
\hspace*{-65pt}
\includegraphics[width=0.45\textwidth]{./figures/T22d1.pdf} &
\hspace*{-65pt}
\includegraphics[width=0.45\textwidth]{./figures/T22d3.pdf} &
\hspace*{-65pt}
\includegraphics[width=0.45\textwidth]{./figures/T22d4.pdf} \\
\hspace*{-65pt}
\includegraphics[width=0.45\textwidth]{./figures/T32d1.pdf} &
\hspace*{-65pt}
\includegraphics[width=0.45\textwidth]{./figures/T32d3.pdf} &
\hspace*{-65pt}
\includegraphics[width=0.45\textwidth]{./figures/T32d4.pdf}
\end{tabular}
\caption{Total sensitivity index for $N_1$ (top row) $N_2$ (center row) and $N_3$ (bottom row)
 at different times as indicated.}
\label{fig:sens2d}
\end{figure}
%%%%%%%%%%%%%%%%%%%%%%%%%%%%%%%%%%%%%%%%%%%%%%%%%%%%%%%%%%%%%%%%
\begin{figure}[h]
\centering

\begin{tabular}{clcl}
\includegraphics[width=0.5\textwidth]{./figures/response_i1_t2.pdf} &
\includegraphics[width=0.5\textwidth]{./figures/response_i2_t2.pdf} \\
\includegraphics[width=0.5\textwidth]{./figures/response_i3_t2.pdf} &
\includegraphics[width=0.5\textwidth]{./figures/response_i4_t2.pdf}
\end{tabular}
\caption{Response surface of water surface elevation at the different gauge locations at t = 7200 s.}
\label{fig:response2}
\end{figure}
%%%%%%%%%%%%%%%%%%%%%%%%%%%%%%%%%%%%%%%%%%%%%%%%%%%%%%%%%%%%%%%%
\begin{figure}[h]
\centering

\begin{tabular}{clcl}
\includegraphics[width=0.5\textwidth]{./figures/response_i1_t3.pdf} &
\includegraphics[width=0.5\textwidth]{./figures/response_i2_t3.pdf} \\
\includegraphics[width=0.5\textwidth]{./figures/response_i3_t3.pdf} &
\includegraphics[width=0.5\textwidth]{./figures/response_i4_t3.pdf}
\end{tabular}
\caption{Response surface of water surface elevation at the different gauge locations at t = 14400 s.}
\label{fig:response3}
\end{figure}
%%%%%%%%%%%%%%%%%%%%%%%%%%%%%%%%%%%%%%%%%%%%%%%%%%%%%%%%%%%%%%%%

\begin{figure}[h]
\begin{tabular}{clc}
%        
\includegraphics[width=0.475\textwidth]{./figures/compare1.pdf} &
\includegraphics[width=0.475\textwidth]{./figures/compare2.pdf} \\
\includegraphics[width=0.475\textwidth]{./figures/compare3.pdf} &
\includegraphics[width=0.475\textwidth]{./figures/compare4.pdf}
\end{tabular}
\caption{Comparison of PC mean 
and observed data of water surface elevation with time at the four gauges.}
\label{fig:compare}
\end{figure}  
%%%%%%%%%%%%%%%%%%%%%%%%%%%%%%%%%%%%%%%%%%%%%%%%%%%%%%%%%%%%%%%%
\begin{figure}[h]
\centering
\includegraphics[width=0.475\textwidth]{./figures/scatter.pdf} 
\caption{Scatter plot of PC mean water surface elevation vs. observed ones.}
\label{fig:scatter}

\end{figure}  
%%%%%%%%%%%%%%%%%%%%%%%%%%%%%%%%%%%%%%%%%%%%%%%%%%%%%%%%%%%%%%%%
\begin{figure}[h]
\begin{tabular}{clc}
\includegraphics[width=0.475\textwidth]{./figures/chain_p1.pdf} &
\includegraphics[width=0.475\textwidth]{./figures/chain_p2.pdf} \\
\includegraphics[width=0.475\textwidth]{./figures/chain_p3.pdf} &
\includegraphics[width=0.475\textwidth]{./figures/chain_s1.pdf}
\end{tabular}
\caption{Chain samples for the three Manning's roughness coefficients $N_1,N_2,N_3$ and $\sigma^2$
the variance between simulations and observations.}
\label{fig:mcmc} 
\end{figure}
%%%%%%%%%%%%%%%%%%%%%%%%%%%%%%%%%%%%%%%%%%%%%%%%%%%%%%%%%%%%%%%%
 \begin{figure}[h]
        \begin{tabular}{clc}
\includegraphics[width=0.475\textwidth]{./figures/pdf_p1.pdf} &
\includegraphics[width=0.475\textwidth]{./figures/pdf_p2.pdf} \\
\includegraphics[width=0.475\textwidth]{./figures/pdf_p3.pdf} &
\includegraphics[width=0.475\textwidth]{./figures/pdf_s1.pdf}
        \end{tabular}
        \caption{Posterior distributions for the three Manning's roughness coefficients $N_1,N_2,N_3$ 
and $\sigma^2$ the variance between simulations and observations.}
\label{fig:pdfs} 
        \end{figure}
 

\clearpage
\bibliographystyle{elsarticle-num}
\bibliography{biblio}
\end{document}
