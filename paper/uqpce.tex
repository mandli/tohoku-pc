%!TEX root = paper.tex
\subsection{Accelerating Bayesian Inference}
\label{sec:uqpce}

The four-dimensional posterior in Equation~\eqref{eq:post_coef} can be directly
explored via MCMC; this requires repeated simulations (tens of thousands) of the forward \geoclaw model, 
once for every proposed set of parameters of the Markov chain~\cite{MarzoukNajm2009,Malinverno2002}. While a single \geoclaw simulation
takes $\sim 15~$mins, depending on the details of the MCMC algorithm used, it is desirable 
to avoid running the forward model at every realization of the MCMC. This is achieved by constructing a 
surrogate model that requires a much
smaller ensemble of \geoclaw runs, and that can be used instead
at a significantly reduced computational cost.  Here, we rely on
Polynomial Chaos expansions for accelerating Bayesian inference in this context 
by building a surrogate model, which, in addition can efficiently
provide statistical properties, such as the mean, variance and sensitivities. 

\subsubsection{Polynomial Chaos}

Polynomial Chaos (PC) is a probabilistic methodology that expresses the 
dependencies of model outputs on the uncertain model inputs
as a polynomial truncated expansion. This method has been developed in 
the engineering community to represent uncertainties in the output of 
numerical simulations~\citep{Villegas2012,Lin2009,Xiu2004}
due to the uncertainties in a model's input. We briefly describe the PC
method below; for more details 
the reader is referred to \citep{LeMaitreKnio2010}.

Let $U=U(\bm{x},t,\xxi)$ denote a quantity of 
interest (QoI) that is the output of a computational model.
$U$ is function of space, $\bm{x}$, and time, $t$, and 
also depends on the canonical vector of random variables $\xxi=(\xi_1,...,\xi_n)$
that parameterize the uncertain inputs. 
PC expresses $U$ in the form:

\begin{equation}
  U(\xbold,t,\xxi) \doteq \sum_{k = 0}^P U_k(\xbold,t) \Psi_k(\xxi),
\label{eq:stochseries}
\end{equation} 
where $U_k(\xbold,t)$ are the polynomial coefficients, and
$\Psi_k(\xxi)$ are functions that form an orthogonal basis of an underlying probability
space. The total number of terms in the PC expansion is
$P+1 = \frac{(d+p)! }{n!\ p!}$ where $n$ is the number of stochastic dimensions and $p$ is the highest order
polynomial order. 

The choice of the basis is dictated by the probability density
function of the stochastic variable $\xxi$, which appears as a weight
function in the probability space's inner product:

\begin{equation}
 \left<\Psi_i,\Psi_j\right> = \int \Psi_i(\xxi) \;\Psi_j(\xxi) \; \rho(\xxi) \; \mbox{d}\xxi=\delta_{ij}\ave{\Psi_i^2},
\label{eq:inner}
\end{equation}
where $\delta_{ij}$ is the Kronecker delta.
For uniform
distributions, the basis functions are scaled Legendre polynomials.
For multi-dimensional problems the basis functions are
tensor products of 1D basis functions~\cite{LeMaitreKnio2010}.

The series representation \eqref{eq:stochseries} can be viewed as a spectral expansion
of $U$ along the stochastic dimensions. It can also be seen as a
combination of approximation and probabilistic frameworks; this
 has proven extremely useful in solving UQ problems~\cite{Xiu:2003,Lin2009}. The existence and convergence of this series is asserted by the Cameron-Martin theorem \citep{Cameron:1947} with the condition that $U$ has a finite variance.
The series rate of convergence, and hence the number of terms to retain, depends on the smoothness of
$U$ with respect to $\xxi$. The series converges spectrally fast with $P$
when $U$ is smooth; the convergence rate becomes algebraic
when $U$ has finite smoothness \citep{Canuto:2006}. In practice the series convergence is monitored 
via various error metrics as discussed in the results section.

\subsubsection{Non Intrusive Spectral Projection (NISP)}
The computation of the coefficients of the PC expansions $U_k$
can be done using a number of procedures. Here we adopt a non-intrusive
approach that allows the use of the forward model \geoclaw as a black box
with no code modifications required. PC expansion coefficients are determined
based on a set of response \geoclaw simulations at a specified set of the uncertain parameters. 
Specifically, we rely on the Non-Intrusive Spectral Projection (NISP) method that exploits the orthogonality of the basis and applies the Galerkin projection to find the PC expansion coefficients as follows:

\begin{equation}
 U_k(\bm{x},t) = \frac{\left< U, \Psi_k \right>}{\left< \Psi_k, \Psi_k \right>} = 
 \frac{1}{\left< \Psi_k, \Psi_k \right>} 
 \int U(\bm{x},t,\xxi) \Psi_k(\xxi) \rho(\xxi) \mbox{ d}\xxi.
\end{equation}
This orthogonal projection minimizes the $L_2$ error on the space spanned by the basis.
Using NISP the stochastic integrals are solved using a numerical quadrature to obtain:

\begin{equation}
  \left< U, \Psi_k \right> 
\approx \left< U, \Psi_k \right>_Q
= \sum_{q=1}^Q U(\xxi_q) \Psi_k(\xxi_q) \omega_q,
\end{equation}
where the subscript $Q$ refers to approximating the inner product integral with
quadrature, and $\xxi_q$ and $\omega_q$ are multi-dimensional quadrature points and weights,
respectively. The quadrature order should be commensurate with the
truncation order, and should be high enough to avoid aliasing artifacts.
The choice of quadrature rule is hence critical to the performance
of the PC (in its NISP version at least).

The computation of the ${U}_k$ can thus be expressed as a matrix-vector product of the form:

\begin{equation} 
 U_k(\bm{x},t)=\sum_q \Pi_{kq} U(\bm{x},t,\xxi_q),\;\;\;
 \Pi_{kq}=\frac{\Psi_k(\xxi_q)\omega_q}{\left< \Psi_k, \Psi_k \right>},
\label{eq:nisp}
\end{equation} 
where $\Pi_{kq}$ is the projection matrix and $U(\bm{x},t,\xxi_q)$ is obtained
from an ensemble of the deterministic model realizations with the uncertain parameters set at
the quadrature value $\xxi_q$. 


\subsubsection{Statistical moments and sensitivity analysis}
The identification of the inner product weight function
with the probability distribution of $\xxi$ simplifies the calculations of statistical moments of $U$. 
Noting that since $\Psi_0(\xxi)$ is a constant that is normalized so that 
$\left<\Psi_0,\Psi_0\right>=1$, the expectation and variance of $U$ can be computed as:

\begin{equation}
 E[U] = \int U \, \rho(\xxi) \, \mbox{d}\xxi=\left< U,\Psi_0\right> = U_0,  
 \label{eq:mean}
\end{equation}
and \begin{equation}
 E[(U-E[U])^2] = \int (U-E[U])^2 \, \rho(\xxi) \, \mbox{d}\xxi=\sum_{k=1}^P U_k^2
 \label{eq:sigma}
\left<\Psi_k,\Psi_k\right>.
\end{equation}

PC representations also enable
efficient global sensitivity analysis that quantify the
contributions of different random input parameters to the variance in the output.
This can be done by computing the so-called {\it total} 
sensitivity index $T_i$ that measures the contribution of
the $i^{th}$ random input to total model variability by
computing the fraction of the total variance due to all the terms in the
PC expansion that involve $\xi_i$~\citep{LeMaitreKnio2010,Crestaux,Sudret}
as follows:

% of random variables from the PC representations or Sobol decomposition~\citep{Sobol:1993,Homma:1996,Sobol:2001}. The total sensitivity index  
%
%To get the total sensitivity corresponding to the uncertain
%input $\xi_i$ we compute the total index:

\begin{equation} \label{eq:T-hard}
   T_i =
         \frac{\displaystyle
               \sum_{k \in K_i} U_k^2 \ave{\Psi_k^2}}
              {\displaystyle\sum_{k = 1}^P U_k^2 \ave{\Psi_k^2}},
\end{equation}
where \[
   K_i = \left\{ k \in \{1, \ldots, P\} :
           \vec{\alpha}^k_i > 0 \right\}
        \]
        and $\vec{\alpha}^k$ is the multi-index associated with $k^{th}$ term in the
PC expansion~\cite{LeMaitreKnio2010}.

%Using Equation~\eqref{eq:T-hard}, the computation of $T_i$ is straightforward.


