\subsection{Inferring Manning Friction Coefficient}
 \label{sec:manning}
 
We seek to infer the Manning coefficient from water surface elevation
data measured at different gauges as shown in Figure~\ref{fig:setup}.
Bayesian inference can be applied directly to infer the uncertain parameters
using Equation~\eqref{eq:post}. In our case, the observation data $\vec d$ 
are the water surface elevation at the gauges;
their model predicted counterparts $\vec G$ is predicted by \geoclaw.
The uncertain parameters $\vec \theta$ will be denoted by $\{N_i\}_{i=1}^3$ for the
three Manning Friction coefficients treated as the unknown parameters. 
In this case  Equation~\eqref{eq:post} can be written as:

\begin{equation} 
p(N_1,N_2,N_3,\sigma^2 | \vec d) 
\propto \frac{1}{\sqrt{2 \pi \sigma^2}} 
 \prod_{i=1}^m  
\exp \left\lbrace \frac{-(d_i - G_i)^2}{2 \sigma^2} \right\rbrace
\ p(\sigma^2)p(N_1)p(N_2) p(N_3)
\label{eq:post_coef}
\end{equation}

To complete the definition of the posterior, we need to choose a proper prior that should be based 
on some \emph{a priori} knowledge about the parameters. In our case, a uniform
prior for the model parameters is assumed as indicated:

\begin{equation} 
p(N_i) = \begin{cases}
		\displaystyle \frac{1}{b_i-a_i} &\text{for~} a_i <  N_i \leq b_i ,  \\
		0 &\text{otherwise}  , 
\end{cases}
\end{equation}
where $ [a_i,b_i]$ denote the parameter ranges defined as.
Regarding the variance, the only information we know 
is that $\sigma^2$ is always positive.
We thus assume a Jeffreys prior \citep{sivia}, expressed as:

\begin{equation} 
p(\sigma^2) =  \begin{cases}
		\displaystyle \frac{1}{\sigma^2} &\text{for~} \sigma^2 > 0,  \\
		0 &\text{otherwise}. 
		\end{cases}
\label{eq:var_pr}
\end{equation}

Inferring the coefficients requires 
sampling the posterior. In general, when the space of the unknown 
parameters is multidimensional, a suitable computational strategy is 
the Markov Chain Monte  Carlo (MCMC) method. 
We rely on an adaptive Metropolis MCMC \citep{Gareth2009,Haario2001} to
sample the posterior distribution accurately and efficiently.


