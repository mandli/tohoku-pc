%!TEX root = paper.tex

\section{Approach}

Our study focuses on quantifying the surface displacement uncertainties associated with the Manning's-N friction coefficient using a polynomial chaos approach as applied to the \tohoku tsunami of 2011.  The forward calculations were generated using \geoclaw, a finite volume wave-propagation numerical model that solves the two-dimensional nonlinear shallow water equations.  In conjunction with 

\subsection{\tohoku Tsunami}

The \tohoku earthquake and tsunami of 2011 has been the subject of numerous studies due to the wealth of observational evidence and severity of the tsunami.  

The $\text{M}_\text{w}$ 9.0 great \tohoku earthquake of 2011 and the accompanying tsunami caused massive damage to the Japanese coastlines in the Sendai region and beyond including damage in Crescent City, CA.  The earthquake magnitude was 

A number of studies have looked at the 

\subsection{\geoclaw}



\geoclaw is Based on work by \cite{MacInnes:2013cr}, \cite{Berger:2011du}, \cite{Berger:2011vi}, \cite{George:2008aa}

\subsection{Observations}

Deep-ocean Assessment and Reporting of Tsunamis (DART) buoy system

gauges 21418, 21413, 21401, 21419 measured maximum wave amplitudes of 1.86, 0.77, 0.66 and 0.54 m respectively







\subsection{Parametric Uncertainty}
Uniform distribution is assumed as the prior for the three manning coefficients.
such that UN = [0.005-0.2]. The three regions are shown in Figure~\ref{fig:setup}(right)

\subsection{Observations}
During this tsunami, different gauges were installed to observed the surface elevation
The locations of theses gauges are shown in Figure~\ref{setup} .
The gauges are used in the inverse problem to infer the manning coefficients.
