%!TEX root = paper.tex
\section{Manning’s $n$ Friction Coefficient} \label{sec:manning}

Manning's $n$ law provides a relationship between the friction coefficient
$C_f$ appearing in the shallow water equations (SWE) governing tsunamis and 
the empirically derived roughness parameter $n$:
\begin{equation}
    C_f = \frac{g n^2}{h^{4/3}},
\label{eq:coef}
\end{equation}
The value of $n$ can vary both spatially and temporally due to changes in the
bottom roughness from wave action and other processes.  For tsunami modeling it
is a common practice to assume a single value of Manning's $n$ for the areas
that are initially below sea-level and sometimes use satellite images and 
land-use data to determine the coefficient in areas above sea-level.  This
process both above and below sea-level has several sources of uncertainty, all
of which can have an impact on the quantities of interest in tsunami modeling.   
In MacInnes \emph{et al.}, for
instance, they found that adopting a uniform value of $n = 0.025$ everywhere
lead to poor performance of the forward model's predicted inundation levels.  To
address this they used $n = 0.035$ in the Sendai plane to account for land-uses 
such as pasture, farmland, and rice paddies.  It is uncertainties such as these 
that we wish to address in a systematic way so that accurate risk assessments 
can be done.

The direct result of the uncertainty in the Manning's $n$ coefficient lead to
uncertainties in the quantities of interest such as the water elevation,
especially in the most crucial area in risk assessments, the near-shore. In 
addition,  the prediction of these quantities of interest are often highly 
sensitive to changes in these uncertain parameters.  We seek to therefore 
quantify the parametric uncertainty due to the uncertainty in the Manning's $n$ 
friction coefficient by characterizing the spatial variability using
iso-baths to define distinct regions in the domain that are characterized by a
single Manning's $n$ coefficient.  In this study three regions were used, an
on-shore region (initially above sea-level), a near-shore region (between
sea-level and the 200 meter iso-bath), and the deep-water region (deeper than
200 meters), whose Manning's $n$ coefficient is denoted by $n_1$, $n_2$, and $n_3$ respectively 
as shown in Figure~\ref{fig:ceofs}. To quantify the uncertainties
with respect to those three parameters we employ Polynomial Chaos (PC) 
expansions as used in \cite{sraj:2013a,sraj:2013b}.  The method and its 
application to our problem is
described in Section~\ref{sec:uqpce}.  To represent the uncertainty in the
Manning's $n$ coefficients an uninformative uniform distribution was assumed as 
the prior with physically relevant values of $n$ sampled from the interval 
$n \in [0.005-0.2]$.

% This Manning's $n$
% coefficient is allowed to vary spatially by defining a set of isobaths that 
% partition the domain into disjoint regions where a single value of $n$ can be 
% used.  The values used for $n$ can be chosen using satellite images and
% land-use data but this is impractical for the values used in regions
% below sea-level or in the case of highly variable land-use regions.  As a
% result, values used for $n$ contain large amounts of uncertainty. 

\section{The \tohoku Tsunami} 
\label{sec:tohoku}

The \tohoku earthquake and tsunami of 2011 have been the subject of a number of
studies due to the wealth of observational evidence and severity of the tsunami.
The earthquake itself had an estimated magnitude of $\text{M}_\text{w}$ 9.0 and
caused massive devastation across Japan. The source of the earthquake (epicenter)
was located approximately 72 km east of \tohoku as indicated 
in Figure~\ref{fig:ceofs}a. In our case study, the
simulation used is based on work done in \cite{MacInnes:2013cr} and modified to 
use the specified variable friction field. Hereafter, we briefly describe the
forward computational model, \geoclaw, the  simulation setup used in the
uncertainty analysis, and finally present the observations from NOAA DART buoys.

\comment{Kyle - I kind of figured the MacInnes paper goes into detail on this
so I am not sure what pertinent information should be included here.  We could
do some more info on the fault itself and some shore maximum inundations but we
really aren't using any of that.}

\subsection{\geoclaw} \label{ssub:geoclaw}
The \geoclaw software package uses a finite volume, wave-propagation approach as
described in \cite{LeVeque:1997eg} to solve the two-dimensional shallow water \
equations:
\begin{equation} \label{eq:swe}
    \begin{aligned}
    &\pd{}{t} h + \pd{}{x} (hu) + \pd{}{y} (hv) = 0, \\
    &\pd{}{t}(hu) + \pd{}{x} \left(hu^2 + \frac{1}{2} g h^2 \right ) + \pd{}{y} (huv) = ~~ fhv - gh \pd{}{x} b - C_f |\vec{u}| hu, \\
    &\pd{}{t} (hv) + \pd{}{x} (huv) + \pd{}{y} \left (hv^2 + \frac{1}{2} gh^2 \right) = -fhu - gh \pd{}{y} b - C_f |\vec{u}| hv,
    \end{aligned}
\end{equation} 
where $h$ is the depth of the water column, $u$ and $v$ the velocities in the 
longitudinal and latitudinal directions respectively, $g$ the acceleration due 
to gravity, $b$ the bathymetry, $f$ the Coriolis parameter, and $C_f$ the bottom 
friction coefficient.  
%As is common in tsunami modeling, $C_f$ is calculated 
%given a Manning's $n$ parameterization such that:
%\begin{equation}
%    C_f = \frac{g n^2}{h^{4/3}},
%\label{eq:coef}
%\end{equation}
%where $n$ is an empirically determined parameter known as Manning's $n$.  

The primary computational kernel in \geoclaw is the evaluation of the solution
to the Riemann problem at each grid cell interface.  The Riemann solver used
includes the ability to handle inundation (wet-dry interfaces), well-balancing,
even when the momentum is non-zero, \alert{and entropy corrections}
\cite{George:2008aa}.

One unique feature that \geoclaw has is the ability to adaptively refine the
grids used for the computation essentially following a region of interest as
time progresses (in this case, the disturbance in the surface height $\eta$).
\geoclaw implements these schemes following \cite{Berger:1984ui,Berger:1998aa}
and is described in detail in \cite{Berger:2011du} in the case of tsunami
modeling.

For the simulations presented, refinement thresholds were used that matched what
was presented in \cite{MacInnes:2013cr}.  As was done there, four levels of
refinement were used staring with a resolution of 1 degree in both the
latitudinal and longitudinal directions down to 0.5' resolution (approximately
900 meters) located around the observation locations.  The tolerance for the
refinement criteria for sea-surface anomaly was 0.02 meters.

\subsection{DART Buoy Observations}

The observations used in this study are from the Deep-ocean Assessment and
Reporting of Tsunamis (DART) buoy system developed and maintained by the
National Oceanic and Atmospheric Administration (NOAA).  The purpose of the
network is to provide early-warning detection and forecasting of tsunami
propagation in the Pacific Ocean \cite{Milburn:1996wm}.  The DART buoys closest
to the earthquake source of the \tohoku tsunami were buoys 21401, 21413, 21418,
and 21419 whose locations are shown in Figure~\ref{fig:setup_buoy_locations} and
whose de-tided water surface elevation data for the event are shown in
Figure~\ref{fig:setup_buoy_data}. The time $t=0$ is set to be the start of the
earthquake and is recorded every 15 seconds during the first few minutes of an
event, otherwise recording every minute when triggered (a tsunami has occurred)
or every 15 minutes when not.  We can see that the primary tsunami
waves generated reach the gauges between 40 minutes and 2 hours after the
earthquake.  The large spikes later in time present in the data at gauge 21418
are data anomalies and not subsequent tsunami waves.

\subsection{Bathymetry and Earthquake Source Model}

The bathymetry used in the simulation is a combination of ETOPO 1' and 4'
accurate bathymetry for the region \cite{Amante:2009ud}.  While it is the case
that the inundation region is the most sensitive to changes in the Manning's $n$
coefficient, the finest bathymetry used in MacInnes \emph{et al.} were
not included in the this study due to the availability of the data and the
insensitivity of the simulation to the finer bathymetry at the DART buoy
locations. The bathymetry of the considered regions (and topography) is 
shown in Figure~\ref{fig:setup}a.   

Two earthquake source models were run to test whether the friction UQ analysis
was sensitive to different rupture models, the first is from Saito et al.
\cite{Saito:2011bh} and is based on an inversion analysis based on dispersive
tsunami simulations.  The second is from Ammon \emph{et al.} \cite{Ammon:2011dm} and
used inversion of the seismic waves to reconstruct the rupture.  These two
models were chosen as they take very different approaches to the reconstruction
of the original fault and were shown to obtain the best un-shifted match to the
observations at the DART buoys for their respective types of reconstruction.  As
discussed in MacInnes \emph{et al.}, the source models used do not include rupture
timing which could have an impact on the accuracy of the simulation.

\comment{Still working on getting the Saito \emph{et al.} source model, seems to do better at the DART buoys.}

\comment{We need to address not using the finest bathymetry better}

% This issue
% is mitigated somewhat by the decreasing sensitivity of water elevation levels
% to friction with deepening water but the near shore region, where the setup for
% inundation occurs, still can be highly sensitive to the friction values chosen.