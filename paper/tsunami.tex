%!TEX root = paper.tex
\section{Manning’s $n$ Friction Coefficient} \label{sec:manning}

Manning's $n$ law provides a relationship between the friction coefficient
\eqref{eq:coef} appearing in the shallow water equations \eqref{eq:swe}
governing tsunamis and the roughness parameter $n$.  This Manning's $n$
coefficient varies spatially  and is dependent on the surface
characteristics of the sea-floor.  A direct measurement of $n$ is therefore
impossible, instead, values are often assigned from tables that are empirically
estimated for common types of land that are determined from satellite images. As
a result, the Manning’s n coefficients used in tsunami models often contain
large amounts of uncertainty.

The uncertainty in the Manning's $n$ coefficient leads to uncertainties in the
predicted quantities of interest such as the water elevations and becomes a
health risk during tsunami events. In addition,  the prediction of these
quantities of interest are often highly sensitive to changes in these
uncertain parameters.  In MacInnes \emph{et al.} a Manning's $n$ coefficient of 0.025
was used except on the Sendai plane where 0.035 was used to account for pasture,
farmland and rice paddies.  In this work, we seek to quantify parametric
uncertainty due to the uncertainty in the Manning's $n$ friction coefficient. To
this end, we characterize $n$ as being contasnt over three different 
regions. The three regions are:  on-shore whose coefficient is denoted by
$n_1$, near- shore  denoted by $n_2$, and deep-water denoted by $n_3$. The three
regions are shown in Figure~\ref{fig:ceofs}. To quantify the uncertainties
with respect to those three parameters we employ Polynomial Chaos (PC) expansions as used in
\cite{sraj:2013a,sraj:2013b}.  The method and its application to our problem is
described in Section~\ref{sec:uqpce}.  To represent the uncertainty in the
Manning's $n$ coefficients an uninformative uniform distribution was assumed as the prior with
$n$  sampled from the interval $n \in [0.005-0.2]$.

\section{The \tohoku Tsunami} 
\label{sec:tohoku}

The \tohoku earthquake and tsunami of 2011 has been the subject of a number of
studies due to the wealth of observational evidence and severity of the tsunami.
The earthquake itself had an estimated magnitude of $\text{M}_\text{w}$ 9.0 and
caused massive devastation across Japan.  In our case study, the
simulation is based on work done in \cite{MacInnes:2013cr} and modified to use
the specified variable friction field. Hereafter, we briefly describe the
forward computational model, \geoclaw, the  simulation setup used in the
uncertainty analysis, and finally present the observations from NOAA DART buoys.
\alert{describe a bit more the tsunami and its geophysical location}.
\subsection{\geoclaw} \label{ssub:geoclaw}
The \geoclaw software package uses a finite volume, wave-propagation approach as described in \cite{LeVeque:1997eg} to solve the two-dimensional shallow water equations:

\begin{equation} \label{eq:swe}
    \begin{aligned}
    &\pd{}{t} h + \pd{}{x} (hu) + \pd{}{y} (hv) = 0, \\
    &\pd{}{t}(hu) + \pd{}{x} \left(hu^2 + \frac{1}{2} g h^2 \right ) + \pd{}{y} (huv) = ~~ fhv - gh \pd{}{x} b - C_f |\vec{u}| hu, \\
    &\pd{}{t} (hv) + \pd{}{x} (huv) + \pd{}{y} \left (hv^2 + \frac{1}{2} gh^2 \right) = -fhu - gh \pd{}{y} b - C_f |\vec{u}| hv,
    \end{aligned}
\end{equation} 
where $h$ is the depth of the water column, $u$ and $v$ the velocities in the 
longitudinal and latitudinal directions respectively, $g$ the acceleration due 
to gravity, $b$ the bathymetry, $f$ the Coriolis parameter, and $C_f$ the bottom 
friction coefficient.  As is common in tsunami modeling, $C_f$ is calculated 
given a Manning's $n$ parameterization such that:

\begin{equation}
    C_f = \frac{g n^2}{h^{4/3}},
\label{eq:coef}
\end{equation}
where $n$ is an empirically determined parameter known as Manning's $n$.  

The primary computational kernel in \geoclaw is the evaluation of the solution to the Riemann problem at each grid cell interface.  The Riemann solver used includes the ability to handle inundation (wet-dry interfaces), well-balancing, even when the momentum is non-zero, \alert{and entropy corrections} \cite{George:2008aa}.

%\subsubsection{Adaptivity} \label{ssub:adaptivity}
One unique feature that \geoclaw has is the ability to adaptively refine the grids used for the computation essentially following a region of interest as time progresses (in this case, the disturbance in the surface height $\eta$).  \geoclaw implements these schemes following \cite{Berger:1984ui,Berger:1998aa} and is described in detail in \cite{Berger:2011du} in the case of tsunami modeling.

For the simulations presented, refinement thresholds were used that matched what was presented in \cite{MacInnes:2013cr}.  As was done there, four levels of refinement were used staring with a resolution of 1 degree in both the latitudinal and longitudinal directions down to 0.5' resolution (approximately 900 meters) located around the observation locations.  The tolerance for the refinement criteria for sea-surface anomaly was 0.02 meters.

\subsection{DART Buoys Observations}

The observations used in this study are from the Deep-ocean Assessment and
Reporting of Tsunamis (DART) buoy system developed and maintained by the
National Oceanic and Atmospheric Administration (NOAA).  The purpose of the
network is to provide early-warning detection and forecasting of tsunami
propagation in the Pacific Ocean \cite{Milburn:1996wm}.  The DART buoys closest
to the earthquake source of the \tohoku tsunami were buoys 21401, 21413, 21418,
and 21419 whose locations are shown in Figure~\ref{fig:setup_buoy_locations} and
whose de-tided water surface elevation data for the event are shown in Figure~\ref{fig:setup_buoy_data}.
The time $t=0$ is set to be the start of the earthquake \comment{frequency of data}.  We can see that the
primary tsunami waves generated reach the gauges between 40 minutes and 2 hours
after the earthquake.  The large spikes later in time present in the data at
gauge 21418 are data anomalies and not subsequent tsunami waves.

\subsection{Bathymetry and Earthquake Source Model}

The bathymetry used in the simulation is a combination of ETOPO 1' and 4'
accurate bathymetry for the region \cite{Amante:2009ud}.  The additional finer
resolution bathymetry data used in MacInnes \emph{et al.} were not used in this
study since the simulations were not sensitive to the finer bathymetry at the
DART buoy locations in the deep-water.

Two earthquake source models were run to test whether the friction UQ analysis
was sensitive to different rupture models, the first is from Saito et al.
\cite{Saito:2011bh} and is based on an inversion analysis based on dispersive
tsunami simulations.  The second is from Ammon \emph{et al.} \cite{Ammon:2011dm} and
used inversion of the seismic waves to reconstruct the rupture.  These two
models were chosen as they take very different approaches to the reconstruction
of the original fault and were shown to obtain the best un-shifted match to the
observations at the DART buoys for their respective types of reconstruction.  As
discussed in MacInnes \emph{et al.}, the source models used do not include rupture
timing and this could have an impact on the accuracy of the simulation.

\comment{Still working on getting the Saito \emph{et al.} source model, seems to do better at the DART buoys.}

