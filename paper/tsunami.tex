%!TEX root = paper.tex

\section{Simulation Setup}

In this section we briefly describe the forward computational model used, \geoclaw, and the case study used in the following uncertainty analysis, the great \tohoku earthquake with observations from NOAA DART buoys.

\subsection{\geoclaw} \label{ssub:geoclaw}

The \geoclaw software package uses a finite volume, wave-propagation approach described in \cite{LeVeque:1997eg} to solve the two-dimensional shallow water equations:

\begin{equation} \label{eq:swe}
    \begin{aligned}
    &\pd{}{t} h + \pd{}{x} (hu) + \pd{}{y} (hv) = 0, \\
    &\pd{}{t}(hu) + \pd{}{x} \left(hu^2 + \frac{1}{2} g h^2 \right ) + \pd{}{y} (huv) = ~~ fhv - gh \pd{}{x} b - C_f |\vec{u}| hu \\
    &\pd{}{t} (hv) + \pd{}{x} (huv) + \pd{}{y} \left (hv^2 + \frac{1}{2} gh^2 \right) = -fhu - gh \pd{}{y} b - C_f |\vec{u}| hv
    \end{aligned}
\end{equation}

where $h$ is the depth of the water column, $u$ and $v$ the velocities in the 
longitudinal and latitudinal directions respectively, $g$ the acceleration due 
to gravity, $b$ the bathymetry, $f$ the Coriolis parameter, and $C_f$ the bottom 
friction coefficient.  As is common in tsunami modeling, $C_f$ is calculated 
given a Manning's $n$ parameterization such that:

\begin{equation}
    C_f = \frac{g n^2}{h^{5/3}}
\end{equation}
where $n$ is an empirically determined parameter.  

The primary computational kernel in \geoclaw is the evaluation of the solution to the Riemann problem at each grid cell interface.  The Riemann solver used includes the ability to handle inundation (wet-dry interfaces), well-balancing even when the momentum is non-zero, and entropy corrections \cite{George:2008aa}.

One unique feature that \geoclaw has is the ability to adaptively refine the grids used for the computation essentially following a region of interest as time progresses (in this case, a tsunami wave).  \geoclaw implements these algorithms following \cite{Berger:1984ui,Berger:1998aa} and is described in detail in \cite{Berger:2011du} in the case of tsunami modeling.

\alert{Need to describe how we handled the AMR in the UQ calculation.}

\subsection{The \tohoku Tsunami} \label{ssub:tohoku}

The \tohoku earthquake and tsunami of 2011 has been the subject of a number of studies due to the wealth of observational evidence and severity of the tsunami.  The earthquake itself had an estimated magnitude of $\text{M}_\text{w}$ 9.0 and caused massive devastation across Japan.  In the case study presented, the simulation is based on work done in \cite{MacInnes:2013cr} and modified to use the specified variable friction field.


\subsubsection{Source Model}

Since the previous study \cite{MacInnes:2013cr} did not find that the source model effected the tsunami wave-forms at the DART buoys, we did not expect there to be much difference between the available models.  After comparing the same analysis with two different source models, namely the models from \cite{Ammon:2011dm} and

\begin{itemize}
    \item Perhaps do one of the tsunami inversion fault models
\end{itemize}

Since the previous study found that the source model used did not impact the off-shore gauge stations, the ones used here from \cite{Ammon:2011dm}, was chose the source model that was commonly available.  Access to the finest bathymetry located near the Japanese coastline was also not included as it primarily effected only the shore observations.

\subsubsection{Bathymetry}
The bathymetry is shown in Figure~\ref{fig:setup} in addition to the topography
of the Pacific Ocean.
\subsubsection{Observations}

The observations used in this study are from the Deep-ocean Assessment and Reporting of Tsunamis (DART) buoy system developed and maintained by the National Oceanic and Atmospheric Administration (NOAA).  The purpose of the network is to provide early-warning detection and forecasting of tsunami propagation in the Pacific Ocean \alert{need citation for this?}.  The DART buoys closest to the earthquake source of the \tohoku tsunami were buoys 21418, 21413, 21401, and 21419 whose locations are shown in Figure~\ref{fig:setup}(left) and whose de-tided data for the event are shown in Figure~\ref{fig:setup}(right).
\subsection{Parametric Uncertainty}
In this work, we seek to quantify parametric uncertainty namely due 
to the uncertainty in the roughness parameter $n$. To this end,
we characterize the roughness $N$ by three different parameters
that are varied in three different regions but are assume
constant in each region. The three regions are as follows: 
on-shore who coefficnet is denoted by $N_1$, near-shore 
denoted by $N_2$, and deep-water denoted by $N_3$.
The three regions are shown in Figure~\ref{fig:ceofs}.
To carry out the uncertainty quantification we have employed polynomial chaos expansions 
as used in \cite{sraj:2013a,sraj:2013b}. The method and its application
to our problem is described in Section~\ref{sec:uqpce}.  
To represent the uncertainty in the Manning's $n$ coefficients a uniform distribution was assumed as the prior where $n$ was sampled in the interval $[0.005-0.2]$.  
