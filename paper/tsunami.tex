%!TEX root = paper.tex

\section{Approach}

The approach taken to quantify the surface displacement uncertainties associated with the Manning's $n$ friction parameterization uses a polynomial chaos approach coupled with a shallow water forward model as applied to a real event, the \tohoku tsunami of 2011.  Here we briefly described the setup of the forward model and the prior of the model used.

\subsection{Forward Model}

The forward model we use is the \geoclaw package, a finite volume wave-propagation class that solves the two-dimensional nonlinear shallow water equations  
\begin{equation} \label{eq:swe}
    \begin{aligned}
    &\pd{}{t} h + \pd{}{x} (hu) + \pd{}{y} (hv) = 0, \\
    &\pd{}{t}(hu) + \pd{}{x} \left(hu^2 + \frac{1}{2} g h^2 \right ) + \pd{}{y} (huv) = ~~ fhv - gh \pd{}{x} b - C_f |\vec{u}| u \\
    &\pd{}{t} (hv) + \pd{}{x} (huv) + \pd{}{y} \left (hv^2 + \frac{1}{2} gh^2 \right) = -fhu - gh \pd{}{y} b - C_f |\vec{u}| v
    \end{aligned}
\end{equation}
where $h$ is the depth of the water column, $u$ and $v$ the velocities in the longitudinal and latitudinal directions respectively, $g$ the acceleration due to gravity, $b$ the bathymetry, $f$ the Coriolis parameter, and $C_f$ the bottom friction coefficient.  As is common, $C_f$ is calculated given a Manning's $n$ parameterization such that
\begin{equation}
    C_f = \frac{g n^2}{h^{5/3}}
\end{equation}
where $n$ is determined from empirical roughness studies.

The input data 


\subsubsection{Source Term Evaluation} \label{ssub:source_term_numerics}

Since the effect of friction on the water column is the primary focus of this study, we give a brief description of the process by which friction and the other source terms are included in \geoclaw.  We can write hyperbolic balance laws such as \eqref{eq:swe} commonly in the form
\begin{equation}
    q_t + f(q)_x + g(q)_y = \Psi(q)
\end{equation}
where $q$ are the conserved quantities, $f(q)$ and $g(q)$ the fluxes in the appropriate directions, and $\Psi(q)$ the source terms.  In the case of \eqref{eq:swe} these can be written
\begin{align}
    q = \begin{bmatrix}
        h \\ hu \\ hv
    \end{bmatrix} & &
    f(q) = \begin{bmatrix}
        hu \\ hu^2 + \frac{1}{2} gh^2 \\ huv
    \end{bmatrix} & &
    g(q) = \begin{bmatrix}
        hv \\ huv \\ hv^2 + \frac{1}{2} gh^2
    \end{bmatrix}
    \Psi(q) = \begin{bmatrix}
        0 \\ fhv - ghb_x - C_f |\vec{u}| u \\ -fhu - ghb_y - C_f |\vec{u}| v
    \end{bmatrix}.
\end{align}

In \geoclaw, the bathymetry source term is handled directly in the Riemann solver step due to well-balancing concerns.  The Coriolis and friction terms are handled via a operator-splitting approach that solves two simpler problems, $q_t + f(q)_x + g(q)_y = 0$ and $q_t = \Psi(q)$, combining the updates to $q$.  The simplest splitting is Godunov splitting in which each system is solved alternately using the full time step $\dt$.  Although this approach is only first-order accurate the observed splitting errors do no dominate the overall error in practice (see \cite{LevequeBook2002} for a more thorough discussion).

Starting with the friction and Coriolis, the bottom friction terms can be evaluated as
\[
    (hu)_t = -C_f hu ~~~~\text{and}~~~~ (hv)_t = -C_f hv
\]
and are solved using a backwards Euler method for computing the loss of momentum so that
\[
    (hu)^{n+1}_{ij} = \frac{(hu)^n_{ij}}{1 + (C_f)_{ij} \dt} ~~~~\text{and}~~~~
    (hv)^{n+1}_{ij} = \frac{(hv)^n_{ij}}{1 + (C_f)_{ij} \dt}
\]
where the drag coefficient is computed using the previous time step's state as
\[
    (C_f)_{ij} = \frac{gn_{ij}^2}{(h^n_{ij})^{-7/3}} \sqrt{[(hu)^n_{ij}]^2 + [(hv)^n_{ij}]^2} \left[1-\left(\frac{h_{\text{break}}}{h^n_{ij}}\right)^{\theta_f} \right]^{\gamma_f / \theta_f}.
\]
The values $h_{\text{break}}$, $\theta_f$, and $\gamma_f$ are constant parameters that define the hybridization between the Chezy and Manning's $n$ formulation from \eqref{eq:friction_term}.  The Coriolis terms
\[
    (hu)_t = -f hv ~~~~\text{and}~~~~ (hv)_t = f hu
\]
are evaluated using a matrix exponential up to the 4th term in the series such that the update becomes
\[
    \begin{bmatrix}
        hu \\ hv     
    \end{bmatrix}^{n+1}_{ij} = 
    \begin{bmatrix}
        hu \\ hv     
    \end{bmatrix}^{n}_{ij} \cdot 
    \begin{bmatrix}
        1 - \frac{1}{2} (f \dt)^2 + \frac{1}{24} (f \dt)^4 & f \dt - \frac{1}{6} (f \dt)^3 \\
        - f \dt + \frac{1}{6} (f \dt)^3 & 1 - \frac{1}{2} (f \dt)^2 + \frac{1}{24} (f \dt)^4
    \end{bmatrix}
\]
where $f = 2 \Omega \sin y$ with $\Omega = 2 \pi / 8.61642\times10^4$ and $y$ the longitudinal coordinate in radians. 


\begin{enumerate}
    \item Finite volume wave-propagation approach
    \item Adaptive mesh refinement
    \item Riemann solver - included inundation and well-balanced properties
    \item Implementation of friction as a source-term splitting
\end{enumerate}


\geoclaw is Based on work by \cite{MacInnes:2013cr}, \cite{Berger:2011du}, \cite{Berger:2011vi}, \cite{George:2008aa}

\subsection{\tohoku Tsunami}

The \tohoku earthquake and tsunami of 2011 has been the subject of a number of studies due to the wealth of observational evidence and severity of the tsunami.  The earthquake itself had an estimated magnitude of $\text{M}_\text{w}$ 9.0 and caused massive devastation across Japan.  In this study we have based our simulation on a recent study that attempted to ascertain which source earthquake model matched the observed data optimally \cite{MacInnes:2013cr}.  

\begin{enumerate}
    \item Note that we do not have the bathymetry used in this paper
    \item Figure out which source we are using
\end{enumerate}

\subsection{Observations}

The observations used in this study are from the Deep-ocean Assessment and Reporting of Tsunamis (DART) buoy system developed and maintained by the National Oceanic and Atmospheric Administration (NOAA).  The purpose of the network is to provide early-warning detection and forecasting of tsunami propagation in the Pacific Ocean \cite{Percival:2011}.  The DART buoys closest to the earthquake source of the \tohoku tsunami were buoys 21418, 21413, 21401, and 21419 whose locations are shown in Figure~\ref{fig:setup} and whose de-tided data for the event are show in Figure~\ref{fig:obs}.

% \begin{figure}[tb]    
%     \centering
%     \begin{subfigure}[b]{0.48\textwidth}
%         \includegraphics[width=\textwidth]{./figures/DART_locations}
%         \caption{}
%         \label{fig:dart_locations}
%     \end{subfigure}
%     \begin{subfigure}[b]{0.48\textwidth}
%         \includegraphics[width=\textwidth]{./figures/DART_data}
%         \caption{}
%         \label{fig:dart_data}
%     \end{subfigure}
%     \caption{Stuff}
%   \end{figure}

\subsection{Parametric Uncertainty}
To represent the uncertainty in the Manning's $n$ coefficients a uniform distribution was assumed as the prior where $n$ was sampled in the interval $[0.005-0.2]$.  The regions

Uniform distribution is assumed as the prior for the three manning coefficients.
such that UN = [0.005-0.2]. The three regions are shown in Figure~\ref{fig:setup}(right)

\subsection{Observations}
During this tsunami, different gauges were installed to observed the surface elevation
The locations of theses gauges are shown in Figure~\ref{setup} .
The gauges are used in the inverse problem to infer the manning coefficients.
