%!TEX root = paper.tex

\section{Tsunami Computational Model and Case Study Descriptions} \label{sec:model_description}

In this section we briefly describe the forward computational model used, \geoclaw, and the case study used in the following UQ analysis, the great \tohoku earthquake with observations from NOAA DART buoys.

\subsection{\geoclaw} \label{ssub:geoclaw}

The \geoclaw software package uses a finite volume, wave-propagation approach described in \cite{LeVeque:1997eg} to solve the two-dimensional shallow water equations
\begin{equation} \label{eq:swe}
    \begin{aligned}
    &\pd{}{t} h + \pd{}{x} (hu) + \pd{}{y} (hv) = 0, \\
    &\pd{}{t}(hu) + \pd{}{x} \left(hu^2 + \frac{1}{2} g h^2 \right ) + \pd{}{y} (huv) = ~~ fhv - gh \pd{}{x} b - C_f |\vec{u}| hu \\
    &\pd{}{t} (hv) + \pd{}{x} (huv) + \pd{}{y} \left (hv^2 + \frac{1}{2} gh^2 \right) = -fhu - gh \pd{}{y} b - C_f |\vec{u}| hv
    \end{aligned}
\end{equation}
where $h$ is the depth of the water column, $u$ and $v$ the velocities in the longitudinal and latitudinal directions respectively, $g$ the acceleration due to gravity, $b$ the bathymetry, $f$ the Coriolis parameter, and $C_f$ the bottom friction coefficient.  As is common in tsunami modeling, $C_f$ is calculated given a Manning's $n$ parameterization such that
\begin{equation}
    C_f = \frac{g n^2}{h^{5/3}}
\end{equation}
where $n$ is an empirically determined parameter.  

The primary computational kernel in \geoclaw is the evaluation of the solution to the Riemann problem at each grid cell interface.  The Riemann solver used includes the ability to handle inundation (wet-dry interfaces), well-balancing even when the momentum is non-zero, and entropy corrections \cite{George:2008aa}.

One unique feature that \geoclaw has is the ability to adaptively refine the grids used for the computation essentially following a region of interest as time progresses (in this case, a tsunami wave).  \geoclaw implements these algorithms following \cite{Berger:1984ui,Berger:1998aa} and is described in detail in \cite{Berger:2011du} in the case of tsunami modeling.

\alert{Need to describe how we handled the AMR in the UQ calculation.}

\subsection{The \tohoku Tsunami} \label{ssub:tohoku}

The \tohoku earthquake and tsunami of 2011 has been the subject of a number of studies due to the wealth of observational evidence and severity of the tsunami.  The earthquake itself had an estimated magnitude of $\text{M}_\text{w}$ 9.0 and caused massive devastation across Japan.  In this study we have based our simulation on a recent study that attempted to ascertain which source earthquake model matched the observed data optimally \cite{MacInnes:2013cr}.  From that study, we have used \alert{SOMETHING} source model and the publicly available bathymetry (the finest bathymetry located near the Japanese coastline was not included here and should not have a large impact on the deep-ocean observations we are utilizing).

\subsection{Observations}

The observations used in this study are from the Deep-ocean Assessment and Reporting of Tsunamis (DART) buoy system developed and maintained by the National Oceanic and Atmospheric Administration (NOAA).  The purpose of the network is to provide early-warning detection and forecasting of tsunami propagation in the Pacific Ocean \cite{Percival:2011}.  The DART buoys closest to the earthquake source of the \tohoku tsunami were buoys 21418, 21413, 21401, and 21419 whose locations are shown in Figure~\ref{fig:setup} and whose de-tided data for the event are show in Figure~\ref{fig:obs}.

% \begin{figure}[tb]    
%     \centering
%     \begin{subfigure}[b]{0.48\textwidth}
%         \includegraphics[width=\textwidth]{./figures/DART_locations}
%         \caption{}
%         \label{fig:dart_locations}
%     \end{subfigure}
%     \begin{subfigure}[b]{0.48\textwidth}
%         \includegraphics[width=\textwidth]{./figures/DART_data}
%         \caption{}
%         \label{fig:dart_data}
%     \end{subfigure}
%     \caption{Stuff}
%   \end{figure}

\subsection{Parametric Uncertainty}
To represent the uncertainty in the Manning's $n$ coefficients a uniform distribution was assumed as the prior where $n$ was sampled in the interval $[0.005-0.2]$.  The regions

Uniform distribution is assumed as the prior for the three manning coefficients.
such that UN = [0.005-0.2]. The three regions are shown in Figure~\ref{fig:setup}(right)
