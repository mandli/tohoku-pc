\section{Discussion and conclusions}
\label{sec:conc}

The present study aimed at estimating Manning's $n$ friction coefficient 
which is crucial in the accurate prediction of water surface elevation in 
tsunami events modeling. We proposed a three-dimensional representation
of Manning's $n$ friction coefficient that was characterized using iso-baths 
to define three distinct regions in the domain and represented each region
with a single Manning's $n$ coefficient. The three regions were as follows:
an on-shore region (initially above sea-level), a near-shore region (between
sea-level and the 200 meter iso-bath), and the deep-water region (deeper than
200 meters) whose Manning's $n$ friction coefficient were denoted by $n_1$, $n_2$, 
and $n_3$ respectively. The estimation relied on a Bayesian
inference approach that sharpens the initial estimates of the three uncertain parameters
using real observations. In our case, we used the DART buoys system data that provides water 
surface elevation data. We specifically used the data collected at four different gauges 
during \tohoku tsunami. To accelerate the Bayesian inference, we relied on the polynomial 
chaos expansions to produce a faithful and efficient surrogate of the forward model \geoclaw. 
The main results of this work were:
\begin{itemize}

\item We built a surrogate model using PC expansions that required 125 forward model runs of \geoclaw.
The surrogate model was tested, validated and proved to be faithful.

\item We used the PC surrogate to quantify the uncertainties in the predicted water surface elevation 
due to the uncertainties in the Manning's $n$ coefficient. To this end, we estimated the mean and the standard deviation of water surface elevation and compared it with the measured data collected at found gauges.

\item A global sensitivity analysis was performed to quantity the contribution of each uncertain parameter to the variance in water surface elevation.  We found that the Manning’s $n$ coefficient at the shore $n_2$ is dominant and
contributes the most to the variance in the water surface elevation compared
to the other two Manning’s $n$ coefficients $n_1$ and $n_3$ for almost the
entire tsunami event.

\item We used Bayesian inference and identified the following MAP
estimates for the three $n$ parameters using MCMC, namely $n_2=0.011$ and
$n_3=0.180$, while for $n_1$ no meaningful MAP value could be determined from the
available data.

\end{itemize}
 
Finally, we note that the present study focused on formulating and estimating a low-dimensional representation
of the Manning's $n$ coefficient using UQ techniques namely Bayesian inference and PC expansions.  A high-dimnensional representation of the Manning's $n$ coefficient would require a large number of forward runs that is computationally expensive. Alternatively, advanced techniques can be implemented to reduce the dimensionality of the problem 
that is the objective of a future study.