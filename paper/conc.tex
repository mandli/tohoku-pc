%!TEX root = paper.tex
\section{Discussion and conclusions}
\label{sec:conc}

The present study aimed at estimating Manning's $n$ friction coefficient  which
plays an important role in the accurate prediction of water surface elevation in
tsunami modeling. We proposed a three-parameter representation of Manning's $n$
friction coefficient that was characterized using iso-baths  to define three
distinct regions, on-shore, near-shore, and deep-water, in the domain and
represented each region with a single Manning's $n$ coefficient.  The estimation
relied on a Bayesian inference approach that sharpens the initial estimates of
the three uncertain parameters using real observations.

In our test case, the \tohoku tsunami, we used four DART buoy gauges that
provide  surface elevation information.  To accelerate the Bayesian inference,
we relied on the polynomial  chaos expansions to produce a faithful and
efficient surrogate of the forward model \geoclaw.  This PC surrogate model was
then used to quantify the uncertainties in the predicted water surface
elevations due to the uncertainties in the Manning's $n$ coefficient.  This
included the mean an standard deviation of water surface elevations which were
also compared to the measured buoy data.  A global sensitivity analysis was also
performed in order to quantify the contribution of each uncertain parameter to
the variance in surface elevation.  It was found that the Manning's $n$
coefficient in the near-shore region $n_2$ contributed the most to the variance
in the surface elevations compared to the other two Manning's $n$ coefficients
in the on-shore and deep-water regions.  Finally, using Bayesian inference, MAP
estimates were found for the three region's $n$ parameters using MCMC.  These
values (excluding $n_1$ where no meaningful MAP value was found), $n_2=0.011$
and $n_3=0.180$, are \comment{unexpected and different from the expected values
used commonly in tsunami modeling.  This may be due to a number of other sources
of uncertainty not taken into account in this analysis.}

\comment{Also add discussion of source differences.}


% The main results of this work were:
% \begin{itemize}

% \item We built a surrogate model using PC expansions that required 125 forward model runs of \geoclaw.
% The surrogate model was tested, validated and proved to be faithful.

% \item We used the PC surrogate to quantify the uncertainties in the predicted water surface elevation 
% due to the uncertainties in the Manning's $n$ coefficient. To this end, we estimated the mean and the standard deviation of water surface elevation and compared it with the measured data collected at found gauges.

% \item A global sensitivity analysis was performed to quantity the contribution of each uncertain parameter to the variance in water surface elevation.  We found that the Manning’s $n$ coefficient at the shore $n_2$ is dominant and
% contributes the most to the variance in the water surface elevation compared
% to the other two Manning’s $n$ coefficients $n_1$ and $n_3$ for almost the
% entire tsunami event.

% \item We used Bayesian inference and identified the following MAP
% estimates for the three $n$ parameters using MCMC, namely $n_2=0.011$ and
% $n_3=0.180$, while for $n_1$ no meaningful MAP value could be determined from the
% available data.

% \end{itemize}
 
Finally, we note that the present study focused on formulating and estimating a
low-dimensional representation of the Manning's $n$ coefficient using UQ
techniques namely Bayesian inference and PC expansions.  A high-dimensional
representation of the Manning's $n$ coefficient would require a large number of
forward runs that is computationally expensive.  Alternative methods can be
implemented to reduce the dimensionality of the problem and will be the
objective of a future study.
