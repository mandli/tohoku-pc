%!TEX root = paper.tex

\section{Introduction}

Tsunamis in the past decade have been responsible for some of the most 
deadly and costly natural disasters ever recorded. Coastal communities 
have faced this hazard by assessing the risk they pose, attempting 
to make informed decisions about the likelihood that such an event would 
occur with cost to life and property.  To do this, computational models of 
tsunami events are often employed to explore multiple different scenarios
to predict the water elevation caused by the tsunamis. The accurate prediction 
of water elevations requires accurate estimation of many model parameters that 
are either measured directly, defined empirically, or inferred from
sparsely collected data. Unfortunately, since tsunami events are relatively rare, 
there is an abundance of uncertainty in the input data for these computational 
models ranging from effects of the domain such as bathymetry and friction parameterizations, 
to the earthquake source.  In this article, we aim to estimate the Manning's n friction
coefficient that plays a key role in predicting the water elevation. 
Our approach is to quantify the uncertainty in the predicted water elevation 
and employ an inverse modeling approach to estimate the Manning's n coefficient 
using real observations of water elevation measured during the \tohoku tsunami as a case study.

Previous work looking into uncertainty in the input for tsunami simulations often has focused on the earthquake source but mainly as a way of constraining the various proposed earthquake scenarios only after the event (see for example \cite{MacInnes:2013cr}).  For instance, Sarri \emph{et al.} recently built a statistical surrogate of a landslide-generated tsunami computer model by using an emulator~\cite{Sarri2012}. The implemented emulator was the Gaussian process that requires using a combination of prior knowledge about the simulator and
appropriate choices of functions and parameters. Their work was presented as a proof-of-concept case study
where they performed basic statistical and sensitivity analysis. In this work we focus on the uncertainty in the friction parameterization used in tsunami simulations.  Changes to the friction parameterization has been shown in the past to lead to significant changes to inundation levels \cite{Myers:2001el,Jakeman:2010hk} but these previous studies were limited in the scope of the changes allowed.  In a recent work, Mayo et el. reformulated a statistical data assimilation method generally used in the estimation of model states to estimate Manning's n coefficient~\cite{Mayo2013}. They used a low-dimensional representations of Manning's n coefficients and recovered it by assimilating water elevation data.

In this work, we aim at making use of the recent developments in Uncertainty Quantification 
(UQ) methods that allows probing the sensitivity of realistic tsunami models 
to uncertain parameters, and inferring those parameters from specific observations. 
In particular, we implement an inverse modeling approach that relied on the Bayesian inference approach.  
We will also rely on Polynomial Chaos (PC) expansions to construct a 
faithful surrogate of the response of the large and complex geophysical 
model simulations.  The construction of the surrogate enables us to efficiently 
implement the Bayesian inference formalism as it dramatically reduces the cost of 
sampling the posterior distribution. 

PC expansions have been developed and applied in the engineering community to quantify uncertainties
in numerical simulations; their principal use is in propagating input
uncertainties through large, complex and non-linear models to compute the
ensuing output uncertainties. Several researchers have been 
investigating the applicability of PC expansions for UQ in oceanic simulations
\citep{thacker2012,ashwanth2010,Alexanderian2012}. Recently,
PC combined with Bayesian inference was used to estimate drag coefficient
during Typhoon Fanapi 2010. The same problem was solved using
the adjoint method and MCMC \cite{sraj:2013a,sraj:2013b}.

In this work, we follow a similar procedure as in \cite{sraj:2013a,sraj:2013b}
to estimate Manning'n n coefficient of friction. 
%We use the widely used Manning's n law to provide a relationship between the friction coefficient and the roughness parameter $n$.  The parameter $n$ is also allowed to vary in space determined by the initial water depth to study effects of deep-water, near-shore, and on-shore sensitivities.  
To carry out the uncertainty quantification we have employed polynomial chaos expansions.  Here the quantity of interest is taken to be the surface elevation $\eta$ and will compared to available NOAA Deep-ocean Assessment and Reporting of Tsunamis (DART) buoys for the \tohoku tsunami.  The forward model runs are performed using \geoclaw following the setup and using one of the source models explored in \cite{MacInnes:2013cr}.

\alert{maybe an outline for this paper}
