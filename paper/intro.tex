%!TEX root = paper.tex

\section{Introduction}
\label{sec:intro}
Tsunamis in the past decade have been responsible for some of the most  deadly
and costly natural disasters ever recorded. Coastal communities  have faced this
hazard by assessing the risk they pose, attempting  to make informed decisions
about the likelihood that such an event would  occur with cost to life and
property.  To do this, computational models of  tsunami events are often
employed to explore multiple different scenarios to predict the water elevation
caused by the tsunamis. The accurate prediction  of water elevations, however, requires
accurate estimation of many model parameters that are either measured directly,
defined empirically, or estimated from a collection of observed data. Unfortunately,
since tsunami events are relatively rare,  there is an abundance of uncertainty
in the input data for these computational  models ranging from effects of the
domain such as bathymetry and friction parameterizations,  to the earthquake
source.  In this article, we aim to estimate the Manning's $n$ friction
coefficient. Our approach quantifies the uncertainty in the predicted water
elevation  and employs an inverse modeling approach to estimate the Manning's
$n$ coefficient using observations of water elevation measured during a tsunami event.

Previous work looking into uncertainty in the input for tsunami simulations
often has focused on the earthquake source but primarily as a way of
constraining the various proposed earthquake scenarios only after the event (see
for example \cite{MacInnes:2013cr}).  A similar approach for landslide-generated
tsunami was presented in Sarri \emph{et al.}, by building a statistical surrogate
using an emulator \cite{Sarri2012}. The implemented emulator was the Gaussian
process that requires using a combination of prior knowledge about the simulator
and appropriate choices of functions and parameters. Their work was presented as
a proof-of-concept case study where they only performed basic statistical and
sensitivity analysis.  Changes to the friction parameterization has been shown in 
the past to lead to significant changes to inundation levels \cite{Myers:2001el,Jakeman:2010hk} 
but these previous studies were limited in the scope of the changes to the parameterization allowed.  
In a recent work, Mayo et el. reformulated a statistical data assimilation method
generally used in the estimation of model states to estimate the Manning's $n$
coefficient~\cite{Mayo2013}. They used a low-dimensional representations of
Manning's $n$ coefficients and recovered it by assimilating synthetic water elevation data.

In this work, we focus our attention on the uncertainty in the friction
parameterization used in tsunami simulations. We present a three-dimensional 
representation of Manning's $n$ coefficient and an efficient method
to estimate this representation using real water elevation data. The method proposed here utilizes recent 
developments in Uncertainty Quantification (UQ) methods that allow 
probing of the sensitivity of realistic tsunami  models to uncertain parameters
without modifying the forward model and inferring those parameters
from a number of observations.  In particular, we implement an inverse modeling
approach that relies on the Bayesian inference technique.  We also employ
Polynomial Chaos (PC) expansions to construct a faithful surrogate model of the
response of the large and complex geophysical model simulations.  The
construction of the surrogate enables us to efficiently implement the Bayesian
inference technique as it dramatically reduces the cost of sampling the
posterior distribution.

PC expansions have been developed and applied in the engineering community to
quantify uncertainties in numerical simulations.  The main advantage of
using PC expansions is the ability of propagating input uncertainties through 
large, complex and non-linear models to compute the corresponding output uncertainties. 
Several researchers have been investigating the applicability of PC expansions for 
UQ in oceanic simulations~\citep{thacker2012,ashwanth2010,Alexanderian2012}. 
Recently, PC was combined with Bayesian inference to estimate the drag coefficient 
using temperature data collected during Typhoon Fanapi 2010. The same problem was 
solved using two different techniques: the adjoint method and Markov Chain Monte Carlo
(MCMC)\cite{sraj:2013a,sraj:2013b}.

In this work, we follow a similar procedure as in \cite{sraj:2013a,sraj:2013b}
to estimate Manning's $n$ coefficient using the \tohoku tsunami
event as a case-study. The quantity of interest is taken to be the surface
elevation $\eta$ and is compared to the available NOAA Deep-ocean Assessment and
Reporting Tsunamis (DART) buoys.  The forward model runs are performed using the
\geoclaw package following the setup in MacInnes \emph{et al.} including the
source models analyzed there \cite{MacInnes:2013cr}.

The structure of this manuscript is as follows. First, we discussing the uncertainties 
of the Manning'g $n$ coefficient to motivate the UQ problem in Section~\ref{sec:manning}
and present a proposed three-dimensional representation of $n$. 
In Section~\ref{sec:tohoku}, we describe our case-study the \tohoku tsunami,
summarize the forward simulations setup and present the DART buoys observational data. 
Section~\ref{sec:formu} describes the two main numerical methods employed in the UQ analysis:
Bayesian inference formalism that brings together observations 
and model results (briefly described in Section~\ref{sec:inference}) 
and PC expansions that were used to build the surrogate model (described
in Section~\ref{sec:uqpce}). Section~\ref{sec:results} describes the main results in 
three subsections: first, we perform error and convergence studies
as evidences that the PC surrogate is a faithful 
representation of the \geoclaw model; second, we use the PC surrogate to 
perform a statistical analysis and compute the  sensitivity of simulated water elevations
to the uncertain input parameters; and third, we present the result of the
posterior distributions obtained using Bayesian inference by MCMC. A discussion of our findings and methodology is finally presented in Section~\ref{sec:conc}.