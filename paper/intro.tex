%!TEX root = paper.tex

\section{Introduction}
\label{sec:intro}
Tsunamis in the past decade have been responsible for some of the most  deadly
and costly natural disasters ever recorded. Coastal communities  have faced this
hazard by assessing the risk they pose, attempting  to make informed decisions
about the likelihood that such an event would  occur with cost to life and
property.  To do this, computational models of  tsunami events are often
employed to explore multiple different scenarios to predict the water elevation
caused by the tsunamis. The accurate prediction  of water elevations requires
accurate estimation of many model parameters that  are either measured directly,
defined empirically, or inferred from sparsely collected data. Unfortunately,
since tsunami events are relatively rare,  there is an abundance of uncertainty
in the input data for these computational  models ranging from effects of the
domain such as bathymetry and friction parameterizations,  to the earthquake
source.  In this article, we aim to estimate the Manning's $n$ friction
coefficient. Our approach quantifies the uncertainty in the predicted water
elevation  and employs an inverse modeling approach to estimate the Manning's
$n$ coefficient using observations of water elevation measured during an event.

Previous work looking into uncertainty in the input for tsunami simulations
often has focused on the earthquake source but primarily as a way of
constraining the various proposed earthquake scenarios only after the event (see
for example \cite{MacInnes:2013cr}).  A similar approach for landslide-generated
tsunami was presented in Sarri \emph{et al.}, building a statistical surrogate
using an emulator \cite{Sarri2012}.  The implemented emulator was the Gaussian
process that requires using a combination of prior knowledge about the simulator
and appropriate choices of functions and parameters. Their work was presented as
a proof-of-concept case study where they performed basic statistical and
sensitivity analysis. In this work we focus on the uncertainty in the friction
parameterization used in tsunami simulations.  Changes to the friction
parameterization has been shown in the past to lead to significant changes to
inundation levels \cite{Myers:2001el,Jakeman:2010hk} but these previous studies
were limited in the scope of the changes to the parameterization allowed.  In
recent work, Mayo et el. reformulated a statistical data assimilation method
generally used in the estimation of model states to estimate the Manning's $n$
coefficient~\cite{Mayo2013}. They used a low-dimensional representations of
Manning's $n$ coefficients and recovered it by assimilating water elevation data.

\alert{Did the Talia et al. study use synthetic data?  Probably should mention
this if that was the case.}

The method proposed here utilizes recent developments in uncertainty
quantification (UQ) methods that allow non-intrusive probing of the sensitivity
of realistic tsunami  models to uncertain parameters, inferring those parameters
from specific  observations.  In particular, we implement an inverse modeling
approach that relies on the Bayesian inference approach.  We will also rely on
Polynomial Chaos (PC) expansions to construct a faithful surrogate of the
response of the large and complex geophysical model simulations.  The
construction of the surrogate enables us to efficiently implement the Bayesian
inference formalism as it dramatically reduces the cost of sampling the
posterior distribution.

PC expansions have been developed and applied in the engineering community to
quantify uncertainties in numerical simulations; their principal use is in
propagating input uncertainties through large, complex and non-linear models to
compute the ensuing output uncertainties. Several researchers have been
investigating the applicability of PC expansions for UQ in oceanic simulations
\citep{thacker2012,ashwanth2010,Alexanderian2012}. Recently, PC combined with
Bayesian inference was used to estimate drag coefficient during Typhoon Fanapi
2010. The same problem was solved using the adjoint method and MCMC
\cite{sraj:2013a,sraj:2013b}.

In this work, we follow a similar procedure as in \cite{sraj:2013a,sraj:2013b}
to estimate Manning's $n$ coefficient of friction using the \tohoku tsunami
event as a case-study.  The quantity of interest is taken to be the surface
elevation $\eta$ and is compared to the available NOAA Deep-ocean Assessment and
Reporting Tsunamis (DART) buoys.  The forward model runs are performed using the
\geoclaw package following the setup in MacInnes \emph{et al.} including the
source models analyzed there \cite{MacInnes:2013cr}.

The structure of this manuscript is as follows. First, we motivate the UQ problem 
by discussing the uncertainties of the drag parameters in Section~\ref{sec:manning}. 
Section~\ref{sec:simulation} summarizes the forward simulations setup
and the DART buoys observational data. 
Section~\ref{sec:formu} describes the two main numerical method employed:
Bayesian inference formalism that brings together observations 
and model results is briefly described in Section~\ref{sec:inference} 
and PC expansions that were used to build the surrogate model are described
in Section~\ref{sec:uqpce}. Section~\ref{sec:results} describes the main results in 
three subsections: first, we present evidence that the surrogate is a faithful 
representation of the \geoclaw model; second, we use the surrogate to explore the 
response surfaces of simulated water elevations, and to compute their sensitivity 
to the control parameters; and third, we present the posterior distribution obtained 
via MCMC sampling and contrast it to the prior distributions. 
A discussion of our findings and methodology is presented in Section~\ref{sec:conc}.