%!TEX root = paper.tex

\section{Introduction} \label{sec:intro}

Tsunamis in the past decade have been responsible for some of the most  deadly
and costly natural disasters ever recorded. Coastal communities  have faced this
hazard by assessing the risk they pose, attempting  to make informed decisions
about the likelihood that such an event would  occur with cost to life and
property.  Toward this goal, computational models of  tsunami events are often
employed to explore multiple different scenarios to predict the water elevation
caused by the tsunamis. The accurate prediction  of water elevations, however,
requires accurate estimation of many model parameters that are either measured
directly, defined empirically, or estimated from a collection of observed data.
Unfortunately, since tsunami events are relatively rare,  there is an abundance
of uncertainty in the input data for these computational  models ranging from
effects of the domain, such as bathymetry and friction parameterizations,  to
the earthquake source.  In this study, we aim to estimate the Manning's $n$
friction coefficient, a parameterization of the effect of bottom friction 
commonly used in tsunami models.  We aim at quantifying the uncertainty in the 
predicted water elevation  and employ
a Bayesian inverse modeling approach to estimate the Manning's $n$ coefficient
using observations of water elevation measured during a tsunami event.

Previous work looking into uncertainty in the input for tsunami simulations
focused on the earthquake source but primarily as a way of
constraining the various proposed earthquake scenarios only after the event (see
for example \cite{MacInnes:2013cr}).  A similar approach for landslide-generated
tsunami was presented in Sarri \emph{et al.}~\cite{Sarri2012}, by building a statistical surrogate
using an emulator. The implemented emulator was based on a Gaussian
process that requires using a combination of prior knowledge about the simulator
and appropriate choices of functions and parameters. Their work was presented as
a proof-of-concept case study where they only performed basic statistical and
sensitivity analysis.  Changes to the friction parameterization has been shown in 
the past to lead to significant changes to inundation levels \cite{Myers:2001el, 
Jakeman:2010hk}, but these previous studies were limited in the range of the 
values and locations used in the parameterization.  
In a recent work, Mayo \emph{et al.}~\cite{Mayo:2014} reformulated a statistical data assimilation method
generally used in the estimation of model states to estimate the Manning's $n$
coefficient. They used a low-dimensional representation of
Manning's $n$ coefficients and recovered it by assimilating synthetic water elevation data
in a storm surge setting.

In this work, we focus our attention on the uncertainty in the friction
parameterization used in tsunami simulations. We present a three-parameter 
representation of Manning's $n$ coefficient and an efficient method
to estimate this representation using real water elevation data. The method proposed here 
is based on recent developments in Uncertainty Quantification (UQ) methods that allow 
probing of the sensitivity of realistic tsunami  models to uncertain parameters
without modifying the forward model and inferring those parameters
from a number of observations.  In particular, we implement an inverse modeling
approach that relies on the Bayesian inference technique.  We resort to
Polynomial Chaos (PC) expansions to construct a faithful surrogate model of the
response of the large and complex geophysical model simulations \cite{Najm:2009,Alexanderian2012,sraj:2013a,sraj:2013b}. The construction of the surrogate enables us to efficiently implement the Bayesian
inference technique as it dramatically reduces the cost of sampling the
posterior distribution.

PC expansions have been developed and applied in the engineering community to
quantify uncertainties in numerical simulations.  The main advantage of
using PC expansions is the ability of propagating input uncertainties through 
large, complex and non-linear models leading to the corresponding output uncertainties. 
Several studies already investigated the efficiency of PC expansions for 
UQ in oceanic simulations~\citep{thacker2012,ashwanth2010,Alexanderian2012}. 
Recently, PC was combined with Bayesian inference to estimate the wind drag coefficient 
using temperature data collected during Typhoon Fanapi 2010. The same problem was 
solved using two different techniques: the adjoint method and Markov Chain Monte Carlo
(MCMC)\cite{sraj:2013a,sraj:2013b}.

In this work, we follow a similar procedure as in \cite{sraj:2013a,sraj:2013b}
to estimate Manning's $n$ coefficient using the \tohoku tsunami
event as a case-study. The quantity of interest is taken to be the water surface
elevation and is compared to the available NOAA Deep-ocean Assessment and
Reporting Tsunamis (DART) buoys.  The forward model runs are performed using the
\geoclaw package following the setup in MacInnes \emph{et al.}~\cite{MacInnes:2013cr} 
including the two source models analyzed there.

The structure of this manuscript is as follows. First, we discuss the uncertainties 
of the Manning'g $n$ coefficient to motivate the UQ problem in Section~\ref{sec:manning}
and propose a three-parameter representation of $n$. 
Section~\ref{sec:tohoku} describes our case-study, the \tohoku tsunami,
summarizes the forward simulations setup, and presents the DART buoys observational data. 
Section~\ref{sec:formu} presents the two main methods employed in the UQ analysis:
Bayesian inference that brings together observations 
and model results (Section~\ref{sec:inference}) 
and PC expansions that are used to build the surrogate model (Section~\ref{sec:uqpce}). Section~\ref{sec:results} describes the main results in 
three subsections: first, we perform error and convergence studies
as evidences that the PC surrogate is a faithful 
representation of the \geoclaw model; second, we use the PC surrogate to 
perform a statistical analysis and compute the  sensitivity of simulated water elevations
to the uncertain input parameters; and third, we present the results of the
posterior distributions obtained using Bayesian inference by MCMC. A discussion of our findings and methodology is finally presented in Section~\ref{sec:conc}.

%\comment{
%\begin{itemize}
%    \item There are two places where we refer to the parameterization as three-dimensional, I am not sure if this is really three-dimensional.  I know that we are using depth contours but the parameterization is two-dimensional.  I have a feeling that readers would take issue with this.
%    \item Ihab: the representation is 3D meaning we used three different values of n in three regions. The same concept was presented in Mayo paper where they did 2D representation.
%    \item I think the order here needs to be revised to talk about the forward model and the equations we are solving first since the Manning's $n$ discussion refers to equations there. 
%    \item Ihab: Our main objective is to show the UQ techniques and how they work. The equations are known and we do not need to stress on them. We can even remove the whole equation or forward model section.
%\end{itemize}
%}
