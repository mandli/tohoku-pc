%!TEX root = paper.tex

\section{Introduction} 

Tsunamis in the past decade have been responsible for some of the largest natural disasters ever recorded.  


In the face of this hazard, risk assessment and forecasting systems have come under renewed scrutiny in order to build for future events.  Work to improve the computational methods underlying these tasks has seen tremendous advances recently but work done to quantify the uncertainties in the input data such as bathymetry, source mechanisms, and friction field, has been lacking.   

Tsunami hazards have come under increasing interest in the past decade as massive tsunami events have turned the world's eyes to the utter destruction they can cause.  Risk assessments and forecasting of these events have seen vast improvements over the past decades due to more readily available computational power and advances in the computational methods employed.  These advances have been responsible for many increases in the veracity of predictive models but one area of tsunami modeling that has not been addressed fully is in the uncertainty of input data such as the bathymetry, source mechanism and friction field.

Previous work to study uncertainty in the input 

In this work we focus on the uncertainty in the parameterization of the friction field.

In this work, we aim to quantify the uncertainty in the friction field using polynomial chaos expansions.  


In this work we aim to quantify one aspect of the uncertainty involved in tsunami modeling, that of the friction field.

In this work, we aim to perform an uncertainty quantification
using Polynomial Chaos Expansion. Quantity of interest is the surface elevation.
The uncertainty in the manning coefficient is investigated.
We assume that the Manning coefficient vary by region. Specifically,
we set three values for the manning coefficient namely, at the land (denoted by N1)
and at the shore (denoted by N2) and below the water (denoted by N3).


\cite{sraj:2013a}
\cite{sraj:2013b}

Background literature on friction coefficient sensitivity:
\begin{itemize}
    \item \cite{Myers:2001el} - Crude friction coefficient sensitivity, found it to be important but not fully explored
    \item \cite{Jakeman:2010hk} - Again a crude (two coefficient) sensitivity analysis.
    \item \cite{Dao:2007hr} - Really not a very good paper, also crude SA.
\end{itemize}

\url{http://arxiv.org/pdf/1203.6297.pdf}
\url{http://hal.inria.fr/docs/00/64/38/78/PDF/RR-7809.pdf}