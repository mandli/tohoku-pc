\subsection{Forward Propagation of Uncertainty}
\label{sec:forward}
We now use the surrogate created effectively using PC to propagate prior 
uncertainty through the forward model. We mainly exploit the surrogate 
to study the statistics of the water surface elevation, and to conduct 
a global sensitivity analysis of the impact of the uncertain input parameters.
We note that the statistical and sensitivity analysis presented below 
is performed using the PC coefficients solely; these were calculated
from an ensemble of 125 \geoclaw simulations.
\subsubsection{Statistical analysis}
The mean of a QoI and its standard deviation is computed
from the PC coefficients as shown in Equation~\ref{eq:mean} and Equation~\ref{eq:sigma}. 
In Figure~\ref{fig:ave} we plot the evolution of
the calculated PC mean water surface elevation along with two standard deviations
bounds at the four gauges as indicated in each panel.  
Note that we only show the evolution after $t=6000~s$ where the uncertainty is significant
reflected in a significant standard deviation as seen earlier from the realizations in Figure~\ref{fig:rlzs}.
We observed that the standard deviation ...... 

The same statistical analysis can be performed for the
entire domain in 2D. Figure~\ref{fig:mean2d}(top row) shows
the PC mean water surface elevation for the considered computational
domain at three different times as indicated in the title of each panel.
The standard deviation is also shown in Figure~\ref{fig:mean2d}(bottom row)
at different times. We clearly notice the propagation of the variance
due to the parametric uncertainty with time
and along the tsunami's reflections. 

        
\subsubsection{Sensitivity analysis}
A global sensitivity analysis is next conducted to quantity the contribution of each
uncertain parameter to the variance in water surface elevation. To this end, we calculate 
the total sensitivity index using the PC coefficients as shown in Equation~\eqref{eq:T-hard}~\citep{Alexanderian2012,Sudret,Crestaux}. The evolution of the total sensitivity index
of each of the uncertain parameters is shown in Figure~\ref{fig:sens} at the four gauges. 
The Manning's roughness coefficient at the shore $N_2$ is clearly dominant and contributes
the most to the variance in the water surface elevation compared to the other two 
Manning's roughness coefficients $N_1$ and $N_3$; this is true for almost the entire simulation time
and at the four gauges. The Manning's roughness coefficient
in the bottom of ocean $N_{3}$ at gauge number 21419 exhibits small sensitivity index 
during the second hour of simulation and Manning's roughness coefficient
at the land $N_1$ appears to be an insignificant contributor
to the variance.

In 2D, the sensitivity analysis shows also that $N_2$ is dominant
where we have variance in water surface elevation. This is
indicated from Figure~\ref{fig:sens2d} that shows the total sensitivity index
for $N_1$(top row) $N_2$(center row) and $N_3$ (bottom row)
at different times as indicated in each panel.

\subsubsection{Response surface}
In addition to the static and analysis, the PC surrogate 
can be used to construct a response surface for the uncertain input parameters.
This is achieved by sampling the PC surrogate for different values of the germ $\xxi$ within the prior
range. Since, $N_3$ shows insignificant contribution in the 
uncertainty in the model output, we find the impact of $N_2$ and $N_3$ for 
fixed $N_1=0.1025$ as illustrated in Figure~\ref{fig:response2}
and Figure~\ref{fig:response3}
for different times as indicated. The most striking features are the relatively flat
horizontal contours in the $N_3$ direction indicating that water surface elevation depends
only mildly on $N_3$ even during peak tsunami events.
